\section{Discounted Cash Flow Analysis}
\label{sect:dcf-analysis}
\begin{enumerate}
\item The \emph{discounted cash flow analysis} starts with the concept of
\emph{net cash flow} (NCF). \defn{Net cash flow} at time \(t\) (denoted by
\(C_t\)) is
\[
\text{(total) cash inflow @ time \(t\)}-\text{(total) cash outflow @ time \(t\)}.
\]
\begin{remark}
\item So, to be more precise, NCF is net cash \emph{inflow}.
\item Sometimes we drop the word ``net'' and just call NCF as ``cash flow'' (like in \cref{sect:int-prob-annuities}).
\item \defn{Discounted cash flow analysis} refers generally to any analysis involving
discounting CFs. We will discuss two kinds of discounted cash flow analysis:
project analysis in \cref{subsect:project-analysis} and fund return analysis in
\cref{subsect:fund-return}.
\end{remark}
\end{enumerate}
\subsection{Project Analysis With Investment Criteria}
\label{subsect:project-analysis}
\begin{enumerate}
\item Here we consider a project with NCFs (at time \(0,1,\dotsc,n\))
\(C_0,C_1,,\dotsc,C_n\) (NCF at time point other than \(0,1,\dotsc,n\) is
zero). (So the project is \emph{terminated} at time \(n\) since from that time
point onward there is no more NCF.)

\begin{note}
Theoretically, a project can \emph{never terminate} (having infinitely many
positive NCFs). But here we shall focus on a project with finitely many
positive NCFs like above.
\end{note}
\item Here we will introduce several investment criteria to \emph{analyze} the
project:
\begin{itemize}
\item net present value (NPV);
\item internal rate of return (IRR) or yield rate;
\item discounted payback period (DPP).
\end{itemize}
\item \defn{Net present value} (NPV) of the project at an interest rate \(i\)
is the PV of the NCFs at rate \(i\):
\[
\text{NPV}=P(i)=\sum_{t=0}^{n}C_tv^t.
\]
The solution to the equation \(P(i)=0\) (unknown: \(i\)) is called
\defn{internal rate of return} (IRR) or \defn{yield rate} of the project.

\begin{note}
``Internal'' suggests that the calculation of IRR itself does not consider
``external factors'' like inflation. (But of course, one may already
incorporate some ``external information'' into the NCFs by ``adjusting'' them
based on some ``external factors'').
\end{note}
\item \label{it:acc-disc-ncf-meaning}
Before proceeding further, let us clarify on the meaning of
accumulating/discounting positive and negative NCF.
\begin{itemize}
\item Accumulating
\begin{itemize}
\item positive NCF: lending/investing that amount for a time length (earning
\emph{lending/investing (interest) rate}) and then collect proceeds from the
loan later \faIcon{arrow-right} becomes a positive NCF at a later time after
accumulation;
\item negative NCF: borrowing/financing that amount (in absolute value) for a
time length (earning \emph{borrowing/financing (interest) rate}) and then repay
the loan later \faIcon{arrow-right} becomes a negative NCF at a later time
after accumulation;
\end{itemize}
\item Discounting
\begin{itemize}
\item positive NCF: deducing NCF at a previous time that would accumulate to
this positive NCF at this time (discount at \emph{lending/investing rate});
\item negative NCF: deducing NCF at a previous time that would accumulate to
this negative NCF at this time (discount at \emph{borrowing/financing rate}).
\end{itemize}
\end{itemize}
\begin{remark}
\item We can see that a minor difference between accumulating/discounting
positive and negative NCFs is that different \emph{kinds} of interest rate are
used.  When the lending and borrowing rates differ, we need to be careful about
the appropriate interest rate to be used. For example, when we calculate NPV,
positive (negative) NCF needs to be discounted at lending (borrowing) rate.
\item Unless stated otherwise, we shall assume that lending and borrowing rates
coincide for convenience.
\end{remark}

\item Now we discuss \emph{practical meanings} of NPV and IRR (which are
important for us to analyze the project using these criteria).

\item \label{it:interpret-npv}
For NPV, let \(i_0\) be the lending (or borrowing) rate.  Then, consider the
following cases:
\begin{itemize}
\item (\(\text{NPV}>0\)) NPV is the amount we need to \emph{invest/lend} now to
accumulate (separately) to yield \emph{all} the NCFs in the project (i.e., to
``get'' all the project NCFs) \faIcon{arrow-right} the project NCFs have a
``positive worth'' as a whole (``like'' an asset).
\item (\(\text{NPV}<0\)) NPV (in absolute value) is the amount we need to
\emph{borrow/finance} now to ``get'' all the project NCFs \faIcon{arrow-right}
the project NCFs have a ``negative worth'' as a whole (``like'' a debt).
\item (\(\text{NPV}=0\)) We do not need to invest anything to ``get'' all the
project NCFs \faIcon{arrow-right} the project NCFs have ``zero worth'' as a
whole. (The project \emph{breaks even}.)
\end{itemize}
\item \label{it:npv-pract-meanings}
Based on the interpretation in \labelcref{it:interpret-npv}, we have the
following practical meanings of NPV:
\begin{itemize}
\item A project is \emph{profitable} (has ``positive worth'') if and only if
its NPV \(>0\).
\item Project A is \emph{more profitable} (``worth more'') than project B if
and only if the NPV of project A \(>\) that of project B.
\end{itemize}
\item For IRR, it is instructive to ``associate'' the project with a ``fund''
\faIcon{piggy-bank} where cash inflow corresponds to \emph{withdrawal}
\faIcon{hand-holding-usd} from the fund, and cash outflow corresponds to
\emph{contribution} \faIcon{donate} to the fund.

Then, for an interest rate \(i\) to qualify as a constant ``rate of return''
for the project, \emph{accumulating} all the withdrawals and contributions at
that interest rate \(i\) to the end of project should give the ``final fund
value''\footnote{Intuitively speaking, we keep accumulating ``remaining
balance'' (even for negative balance; may be seen as ``accumulating overdraft
charges'' in this case) in the fund \faIcon{piggy-bank}. To interpret the
accumulation more formally, one may convert ``withdrawals'' and
``contributions'' to ``positive NCFs'' and ``negative NCFs'' respectively, then
refer to \labelcref{it:acc-disc-ncf-meaning}.} , i.e.,
\begin{itemize}
\item (when the value is positive) amount we get if we withdraw all money from
the fund, or
\item (when the value is negative) amount we need to further contribute to the
fund for ``clearing the debt''.
\end{itemize}
(In either case, the value is equal to the time-\(n\) NCF \(C_n\).)
In other words, the rate \(i\) should satisfy
\[
\sum_{t=0}^{n-1}C_t(1+i)^{n-t}=C_n \iff P(i)=0 \iff \text{\(i\) is IRR}.
\]
Hence, IRR can be interpreted as a \emph{possible} constant rate of return for
the project, as one may expect.

\begin{remark}
\item Note that the return rate is only ``possible'': The project may not even
have constant rate of return!
\item Sometimes there are multiple IRRs, indicating that there are several
\emph{possible} constant rates of return for the project, just by observing
the NCFs.
\end{remark}
\item A ``typical'' NPV graph looks like the following --- a strictly
decreasing function in \(i\) which crosses \(x\)-axis exactly once:

\begin{intuition}
Usually this happens when the early CFs are relatively more negative, and the
later CFs are relatively more positive (which are ``typical''). Then, as the
interest rate \(i\) rises, those relatively more positive CFs get discounted
more heavily \faIcon{arrow-right} reducing NPV.
\end{intuition}
\begin{center}
\begin{tikzpicture}
\begin{axis}[domain=0:5.5, ymax=2, samples=100, axis lines=middle,
ylabel={NPV}, xlabel={\(i\)},
ylabel style={at={(axis description cs:0,1)}, anchor=south},
xlabel style={anchor=west},
xtick={1.2247}, xticklabel={IRR}, ytick=\empty
]
\addplot[blue]{-1+2/(1+x)+0.5/(1+x)^2};
\draw[opacity=0.4, green, line width=0.2cm] (0,0) -- (1.2247,0)
node[pos=0.6, above, ForestGreen, opacity=1]{profitable};
\end{axis}
\end{tikzpicture}
\end{center}
Under this ``typical'' case, the project is profitable (i.e., NPV \(>0\)) if
and only if the interest rate \(i<\text{IRR}\), giving a profitability
criterion based on IRR.
\item However, unlike NPV, IRR \emph{cannot} be used for comparing
profitability of different projects \warn. To see this, consider the following
graph:
\begin{center}
\begin{tikzpicture}
\begin{axis}[domain=0:5.5, ymax=2, samples=100, axis lines=middle,
ylabel={NPV}, xlabel={\(i\)},
ylabel style={at={(axis description cs:0,1)}, anchor=south},
xlabel style={anchor=west},
xtick={1.2247, 1.762}, xticklabels={\(i_A\), \(i_B\)}, ytick=\empty,
legend entries={project A, project B}
]
\addplot[blue]{-1+2/(1+x)+0.5/(1+x)^2};
\addplot[orange]{-0.5+1.2/(1+x)+0.5/(1+x)^2};
\draw[opacity=0.4, green, line width=0.2cm] (0,0) -- (0.6,0);
\draw[opacity=0.3, violet, line width=0.2cm] (0.6,0) -- (5.4,0);
\draw[-Latex, ForestGreen] (1,-0.4) -- (0.3,-0.1)
node[pos=-0.3]{A more profitable};
\draw[-Latex, violet] (3.5,0.4) -- (2.5,0.1)
node[pos=-0.3]{B more profitable};
\end{axis}
\end{tikzpicture}
\end{center}
As seen in the graph, IRR of project B (\(i_B\)) is greater than IRR of project
A (\(i_A\)), but project A is more profitable than project B for some \(i\)!
\item In some ``non-typical'' case, the NPV graph can look weird and it can
have \emph{multiple roots} or even \emph{no root}! \faIcon{arrow-right} IRR may
\emph{not} be unique and may \emph{not} exist!
\begin{center}
\begin{tikzpicture}
\begin{axis}[domain=0:5.5, samples=100, axis lines=middle,
ylabel={NPV}, xlabel={\(i\)},
ylabel style={at={(axis description cs:0,1)}, anchor=south},
xlabel style={anchor=west},
xtick=\empty, ytick=\empty,
title={Two IRRs}
]
\addplot[blue]{0.7-1.8/(1+x)-1.5/(1+x)^2+3.3/(1+x)^3};
node[pos=0.6, above, ForestGreen, opacity=1]{profitable};
\end{axis}
\end{tikzpicture}
\begin{tikzpicture}
\begin{axis}[domain=0:5.5, ymin=-0.5, samples=100, axis lines=middle,
ylabel={NPV}, xlabel={\(i\)},
ylabel style={at={(axis description cs:0,1)}, anchor=south},
xlabel style={anchor=west},
xtick=\empty, ytick=\empty,
title={No IRR}
]
\addplot[blue]{0.7-0.3/(1+x)-0.5/(1+x)^2+0.2/(1+x)^3};
\end{axis}
\end{tikzpicture}
\end{center}
\item \label{it:suff-cond-irr-unique}
Now we are interested in knowing when IRR exists uniquely. Here we give a
sufficient condition for its uniqueness, based on \emph{Descartes' rule of
signs}: Consider the project NCFs \(C_0,C_1,\dotsc,C_n\) as a sequence (in this
order).  Then, IRR of the project exists \emph{uniquely} if there is
\emph{exactly one} sign change in the sequence (ignoring any zero NCF in the
sequence, if exist).

\begin{note}
This means when the sequence of NCFs (ignoring zero NCFs; in terms of sign) is
of the form ``\(+ + \cdots + - - \cdots -\)'' or ``\(- - \cdots - + + \cdots
+\)'', then IRR is guaranteed to exist uniquely.
\end{note}
\item \label{it:dpp-def}
Now we discuss the last investment criteria here: \emph{discounted
payback period}. The \defn{discounted payback period} (DPP) of the project at
an interest rate \(i\) is
\[
\text{DPP}=\min\{m\in\N_0:\text{NPV}_{m}\ge 0\}
\]
where \(\displaystyle \text{NPV}_{m}=\sum_{t=0}^{m}C_tv^t\) is the \defn{partial NPV up to
time \(m\)}.
\item DPP can be regarded as the time where the project \emph{first} ``pays
back'' or ``breaks even''. If \(\text{NPV}_{n}<0\) for any \(n\in\N_0\), then
DPP \emph{does not exist}.
\item After removing ``\(v^t\)'' in \labelcref{it:dpp-def}, the resulting time
becomes \defn{payback period} of the project at the interest rate \(i\).
\end{enumerate}
\subsection{Reinvestment Rates}
\begin{enumerate}
\item In \cref{subsect:project-analysis}, the investment (or lending) rate is
assumed to stay constant throughout the project. But this is not necessarily
the case in practice: While \emph{initial investment} may ``capture good
opportunities'' and hence earn a relatively high rate, such ``good
opportunities'' may be already gone for later \emph{reinvestments} of interests,
so those reinvestments may earn a \emph{lower} rate (\emph{reinvestment rate}).
\item Here we will derive some formulas with reinvestment rate \(j\) being distinct
from the initial investment (interest) rate \(i\). To be more precise,
\begin{itemize}
\item the initial investment earns interests (payable at the end of each
period) at the rate \(i\);
\item interests paid are reinvested (immediately) at the rate \(j\) (earning
further interests).
\end{itemize}
\item \label{it:1-acc-value-reinvest}
In this case, the accumulated value of 1 at time \(n\) is
\[
{\color{magenta}1}+{\color{orange}i\sx{\angl{n}j}}.
\]
\begin{center}
\begin{tikzpicture}
\draw[-Latex] (0,0) -- (12,0);
\draw[fill] (0,0) circle [radius=0.5mm];
\draw[fill] (2,0) circle [radius=0.5mm];
\draw[fill] (4,0) circle [radius=0.5mm];
\draw[fill] (8,0) circle [radius=0.5mm];
\draw[fill] (10,0) circle [radius=0.5mm];
\node[] () at (0,-0.5) {0};
\node[] () at (2,-0.5) {1};
\node[] () at (4,-0.5) {2};
\node[] () at (6,-0.5) {\(\cdots\)};
\node[] () at (8,-0.5) {\(n-1\)};
\node[] () at (10,-0.5) {\(n\)};
\node[] () at (0,0.5) {\faIcon{arrow-up}1};
\node[brown] () at (0,1) {\faIcon{piggy-bank}};
\node[] () at (2,0.5) {\faIcon{arrow-up}\(i\)};
\node[violet] () at (2,1) {\faIcon{piggy-bank}};
\node[] () at (4,0.5) {\faIcon{arrow-up}\(i\)};
\node[violet] () at (4,1) {\faIcon{piggy-bank}};
\node[] () at (8,0.5) {\faIcon{arrow-up}\(i\)};
\node[violet] () at (8,1) {\faIcon{piggy-bank}};
\node[] () at (10,0.5) {\faIcon{arrow-up}\(i\)};
\node[violet] () at (10,1) {\faIcon{piggy-bank}};
\node[] () at (10,-1) {{\color{brown}\faIcon{piggy-bank}}{\color{violet}\faIcon{piggy-bank}}};
\node[] () at (10,-1.5) {\faIcon{hand-holding}};
\draw[-Latex, ForestGreen] (0,1.1) to[bend left] (2.2,0.7);
\draw[-Latex, ForestGreen] (0,1.1) to[bend left] (4.2,0.7);
\draw[-Latex, ForestGreen] (0,1.1) to[bend left] (6.2,0.7);
\draw[-Latex, ForestGreen] (0,1.1) to[bend left] (8.2,0.7);
\draw[-Latex, ForestGreen] (0,1.1) to[bend left] (10.2,0.7);
\node[ForestGreen] () at (5,3) {earn rate \(i\)};
\draw[-Latex, magenta] (0,0.8) to[bend right] node[bend right, auto, swap]{no more interest} (9.8,-1.3);
\draw[-Latex, orange] (2.1,0.7) to[bend left] (10.2,-0.8);
\draw[-Latex, orange] (4.1,0.7) to[bend left] (10.2,-0.8);
\draw[-Latex, orange] (6.1,0.7) to[bend left] (10.2,-0.8);
\draw[-Latex, orange] (8.1,0.7) to[bend left] (10.2,-0.8);
\draw[-Latex, orange] (10.1,0.7) to[bend left] (10.2,-0.8);
\node[magenta] () at (9.3,-1) {1};
\node[orange] () at (11,-1) {\(i\sx{\angl{n}j}\)};
\end{tikzpicture}
\end{center}
\item \label{it:annu-imm-acc-value-reinvest}
The accumulated value of an \(n\)-period annuity-immediate (amount of
each CF = 1) at time \(n\) is
\[
{\color{magenta}n}+{\color{orange}i\Is_{\angl{n-1}j}}.
\]
\begin{center}
\begin{tikzpicture}
\draw[-Latex] (0,0) -- (12,0);
\draw[fill] (0,0) circle [radius=0.5mm];
\draw[fill] (2,0) circle [radius=0.5mm];
\draw[fill] (4,0) circle [radius=0.5mm];
\draw[fill] (8,0) circle [radius=0.5mm];
\draw[fill] (10,0) circle [radius=0.5mm];
\node[] () at (0,-0.5) {0};
\node[] () at (2,-0.5) {1};
\node[] () at (4,-0.5) {2};
\node[] () at (6,-0.5) {\(\cdots\)};
\node[] () at (8,-0.5) {\(n-1\)};
\node[] () at (10,-0.5) {\(n\)};
\node[] () at (4,0.5) {\faIcon{arrow-up}\(i\)};
\node[violet] () at (4,1) {\faIcon{piggy-bank}};
\node[] () at (8,0.5) {\faIcon{arrow-up}\((n-2)i\)};
\node[violet] () at (8,1) {\faIcon{piggy-bank}};
\node[] () at (10,0.5) {\faIcon{arrow-up}\((n-1)i\)};
\node[violet] () at (10,1) {\faIcon{piggy-bank}};

\node[] () at (10,-3) {{\color{brown}\faIcon{piggy-bank}}{\color{violet}\faIcon{piggy-bank}}};
\node[] () at (10,-3.5) {\faIcon{hand-holding}};

\node[] () at (2,-1) {\faIcon{arrow-down}1};
\node[brown] () at (2,-1.5) {\faIcon{piggy-bank}};
\node[] () at (4,-1) {\faIcon{arrow-down}1};
\node[brown] () at (4,-1.5) {\faIcon{piggy-bank}};
\node[] () at (8,-1) {\faIcon{arrow-down}1};
\node[brown] () at (8,-1.5) {\faIcon{piggy-bank}};
\node[] () at (10,-1) {\faIcon{arrow-down}1};
\node[brown] () at (10,-1.5) {\faIcon{piggy-bank}};

\draw[-Latex, magenta] (2,-1.7) to[bend right] node[bend right, auto, swap]{no more interest} (9.6,-3);
\draw[-Latex, magenta] (4,-1.7) to[bend right] (9.6,-3);
\draw[-Latex, magenta] (8,-1.7) to[bend right] (9.6,-3);
\draw[-Latex, magenta] (10,-1.7) to[bend right] (9.6,-3);

\draw[-Latex, orange] (4.1,0.7) to[bend left] (10.2,-2.8);
\draw[-Latex, orange] (6.1,0.7) to[bend left] (10.2,-2.8);
\draw[-Latex, orange] (8.1,0.7) to[bend left] (10.2,-2.8);
\draw[-Latex, orange] (10.1,0.7) to[bend left] (10.2,-2.8);
\node[magenta] () at (9,-3) {\(n\)};
\node[orange] () at (11.5,-3) {\(i\Is_{\angl{n-1}j}\)};
\end{tikzpicture}
\end{center}
\begin{note}
The interest payment at time \(m\) is
\[
i\times\text{amount in {\color{brown}\faIcon{piggy-bank}} at time \(m-1\)}
\]
(which is to be reinvested into {\color{violet}\faIcon{piggy-bank}} immediately).
\end{note}
\end{enumerate}
\subsection{Measures of Fund Return Rate}
\label{subsect:fund-return}
\begin{enumerate}
\item Consider here a fund \faIcon{piggy-bank} where we invest money
\faIcon{dollar-sign} in \faIcon{piggy-bank} at time 0 and ``clear'' the balance
in \faIcon{piggy-bank} at time 1\footnote{i.e., withdraw \faIcon{dollar-sign}
when the balance is positive, or contribute \faIcon{dollar-sign} when the
balance is negative (to make the fund balance zero)}, with possibly additional
contributions/withdrawals between time 0 and 1.

\item We introduce the following notations:
\begin{itemize}
\item \(A\): amount in \faIcon{piggy-bank} at time 0 \faIcon{arrow-right} we invest \(A\) into \faIcon{piggy-bank} at time 0
\item \(B\): amount in \faIcon{piggy-bank} at time 1 (just before the
``clearing'') \faIcon{arrow-right} we withdraw \(B\) out of \faIcon{piggy-bank}
at time 1\footnote{Here withdrawing a negative amount means contributing that
amount (in absolute value) into \faIcon{piggy-bank}.}
\item \(I\): amount of interest earned between time 0 and 1
\item \(C_t\): net contribution to \faIcon{piggy-bank} (net cash
\underline{out}flow from our perspective) at time \(t\), where \(0<t<1\) (finitely many)
\begin{note}
Here we regard withdrawal (cash \underline{in}flow) as negative contribution to
\faIcon{piggy-bank}. (Note that contribution is cash \underline{out}flow.)
\end{note}
\begin{warning}
In this context, \(C_t\) is \underline{no longer} NCF at time \(t\)! Indeed, it
is \emph{negative} of NCF at time \(t\)!
\end{warning}
\item \(C\): total net contribution to \faIcon{piggy-bank} in time interval
\((0,1)\): \(C=\sum_{t}^{}C_t\)
\item \(\actsymb[{\color{violet}a}]{i}{{\color{brown}b}}\): amount of interest
earned between time \({\color{violet}a}\) and
\({\color{violet}a}+{\color{brown}b}\), when 1 is invested into \faIcon{piggy-bank}
at time \(a\)
\end{itemize}
\begin{center}
\begin{tikzpicture}
\draw[-Latex] (0,0) -- (12,0);
\draw[fill] (0,0) circle [radius=0.5mm];
\draw[fill] (3,0) circle [radius=0.5mm];
\draw[fill] (6,0) circle [radius=0.5mm];
\draw[fill] (10,0) circle [radius=0.5mm];
\node[] () at (0,-0.5) {0};
\node[] () at (3,-0.5) {\(t_1\)};
\node[] () at (6,-0.5) {\(t_2\)};
\node[] () at (10,-0.5) {1};
\node[] () at (0,0.5) {\faIcon{arrow-up}\(A\)};
\node[] () at (0,1) {\faIcon{piggy-bank}};
\node[] () at (3,0.5) {\faIcon{arrow-up}1};
\node[] () at (3,1) {\faIcon{piggy-bank}};
\node[] () at (3,-1) {\(C_{t_1}=+1\)};
\node[] () at (3,-1.5) {\(\text{NCF}=-1\)};
\node[] () at (6,0.5) {\faIcon{arrow-down}1};
\node[] () at (6,1) {\faIcon{piggy-bank}};
\node[] () at (6,-1) {\(C_{t_2}=-1\)};
\node[] () at (6,-1.5) {\(\text{NCF}=+1\)};
\node[] () at (10,0.5) {\faIcon{arrow-down}\(B\)};
\node[] () at (10,1) {\faIcon{piggy-bank}};
\end{tikzpicture}

\begin{tikzpicture}
\draw[-Latex] (0,0) -- (12,0);
\draw[fill] (0,0) circle [radius=0.5mm];
\draw[fill] (3,0) circle [radius=0.5mm];
\draw[fill] (6,0) circle [radius=0.5mm];
\draw[fill] (10,0) circle [radius=0.5mm];
\node[] () at (0,-0.5) {0};
\node[violet] () at (3,-0.5) {\(a\)};
\node[] () at (6,-0.5) {\({\color{violet}a}+{\color{brown}b}\)};
\node[] () at (10,-0.5) {1};
\node[] () at (3,0.5) {\faIcon{arrow-up}1};
\node[] () at (3,1) {\faIcon{piggy-bank}};
\node[] () at (6,0.5) {\faIcon{coins}: \(1+\actsymb[a]{i}{b}\)};
\node[] () at (6,1) {\faIcon{piggy-bank}};
\end{tikzpicture}
\end{center}
\item Here we discuss the following measures of fund return rate:
\begin{itemize}
\item IRR
\item dollar-weighted rate of return/interest (DWRR)
\item ``simplified'' DWRR
\item time-weighted rate of return/interest (TWRR)
\end{itemize}
\item Let \(i\) be the IRR. Then, by definition we have
\begin{align}
\nonumber&&\text{NPV @ \(i\)}&=0 \\
\nonumber&\Leftrightarrow&
-A+\sum_{t}^{}({\color{Maroon}-}C_t)v^t+Bv&=0 \\
\label{eq:fund-return-eq}
&\Leftrightarrow&
B&=A(1+i)+\sum_{t}^{}C_t\underbrace{(1+i)^{1-t}}_{1+\actsymb[t]{i}{1-t}}.
\end{align}
So, IRR is an interest rate \(i\) such that \cref{eq:fund-return-eq} holds.

\item \label{it:dwrr-fmla}
Now consider the DWRR. First we shall assume \emph{simple interest}
temporarily here. Then, \cref{eq:fund-return-eq} becomes
\[
B=A(1+i)+\sum_{t}^{}C_t\underbrace{(1+(1-t)i)}_{1+\actsymb[t]{i}{1-t}},
\]
which implies
\begin{equation}
\label{eq:fund-return-eq-simple-int}
B-A-C=i\qty[A+\sum_{t}^{}C_t(1-t)].
\end{equation}
\begin{note}
Recall that \(C=\sum_{t}^{}C_t\).
\end{note}

Since changes in the amount in the fund \faIcon{piggy-bank} are only due to the
net contributions and interest, we have \(B=A+\underbrace{C}_{\text{total net
contrib.}}+\underbrace{I}_{\text{interest}}\), and hence can rewrite
\cref{eq:fund-return-eq-simple-int} as
\[
i=\boxed{\frac{I}{A+\sum_{t}^{}C_t(1-t)}}.
\]
This interest rate \(i\) is called \defn{dollar-weighted rate of return}
(DWRR), sometimes denoted by \(i^{\text{DW}}\).

\begin{remark}
\item From this, we see that DWRR is essentially IRR but assuming simple interest.
\item The rate is ``dollar-weighted'' in the sense that the net contributions
(``dollars'' \faIcon{dollar-sign}) influence the rate \(i\) (contribute to
``weight'' for \(i\) in \cref{eq:fund-return-eq-simple-int}).
\end{remark}

\item \label{it:simplified-dwrr-fmla}
Now, for ``simplified'' DWRR, it is simplified from DWRR in the sense
that it assumes contributions/withdrawals can only possibly occur at time 0.5
(``middle'')\footnote{It may be deemed ``reasonable'' when the net
contributions spread quite ``uniformly'' \faIcon{arrow-right} we ``take
average'' in some sense.}. Due to this assumption, we have
\[
\sum_{t}^{}C_t(1-t)=0.5C_{0.5}=0.5C.
\]
Hence, \cref{eq:fund-return-eq-simple-int} becomes
\[
i=\boxed{\frac{2I}{A+B-I}}
\]
and this interest rate \(i\) is \defn{``simplified'' DWRR} (this name is nonstandard).

\begin{note}
The expression is \emph{independent} from \(C\), which is a remarkable property
for ``simplified'' DWRR.
\end{note}
\item The final measure to be discussed is TWRR. For this one, it is \emph{not}
related to IRR, unlike (simplified) DWRR. To understand TWRR, consider the
following graph:
\begin{center}
\begin{tikzpicture}
\begin{axis}[domain=0:5, samples=100,
xmin=-0.2, xmax=5.5, ymin=-0.5, ymax=3.5,
xtick={0,1,2.5,5}, xticklabels={0,\(t_1\),\(t_2\),1},
ytick=\empty, ylabel={Amount in \faIcon{piggy-bank}}, xlabel={Time},
axis lines=middle,
ylabel style={at={(axis description cs:0,1)}, anchor=south},
xlabel style={anchor=west}
]
\addplot[blue, domain=0:0.99]{1+0.5*x^2};
\addplot[opacity=0.4, violet, line width=0.2cm, domain=0:0.99]{1+0.5*x^2};
\draw[blue] (1,1.5) circle [radius=0.5mm];
\draw[blue, fill] (1,1.2) circle [radius=0.5mm];
\draw[-Latex, red] (1,1.5) -- (1,1.2)
node[midway, right]{withdraw};
\addplot[blue, domain=1:2.49]{1.2+0.5*(x-1)^2};
\addplot[opacity=0.4, brown, line width=0.2cm, domain=1:2.49]{1.2+0.5*(x-1)^2};
\draw[blue] (2.5,2.325) circle [radius=0.5mm];
\draw[blue, fill] (2.5,3.2) circle [radius=0.5mm];
\draw[-Latex, ForestGreen] (2.5,2.325) -- (2.5,3.2)
node[midway, left]{contribute};
\addplot[blue, domain=2.5:5]{3.2-0.3*(x-2.5)^2};
\addplot[opacity=0.4, magenta, line width=0.2cm,  domain=2.5:5]{3.2-0.3*(x-2.5)^2};
\node[violet] () at (0.2,0.8) {\(B_0\)};
\node[violet] () at (1.3,1.6) {\(B_1\)};
\node[brown] () at (1.4,1) {\(B_1+C_1\)};
\node[brown] () at (2.8,2.3) {\(B_2\)};
\node[magenta] () at (1.8,3.2) {\(B_2+C_2\)};
\node[magenta] () at (5,1.1) {\(B_3\)};
\draw[opacity=0.4, violet, line width=0.2cm] (0,0) -- (1,0);
\draw[opacity=0.4, brown, line width=0.2cm] (1,0) -- (2.5,0);
\draw[opacity=0.4, magenta, line width=0.2cm] (2.5,0) -- (5,0);
\end{axis}
\end{tikzpicture}
\end{center}
Here we let \({\color{violet}j_1},{\color{brown}j_2},{\color{magenta}j_3}\) be
the effective rate for the (whole) time intervals
\({\color{violet}[0,t_1]},{\color{brown}[t_1,t_2]},{\color{magenta}[t_2,1]}\)
respectively. Then,
\[
{\color{violet}B_0(1+j_1)=B_1},\quad
{\color{brown}(B_1+C_1)(1+j_2)=B_2},\quad
{\color{magenta}(B_2+C_2)(1+j_3)=B_3}.
\]
In this case, the TWRR is
\[
i^{\text{TW}}=(1+j_1)(1+j_2)(1+j_3),
\]
which is ``time-weighted'' in the sense that the quantity only depends on the
``inherent growth rate'' of fund \faIcon{piggy-bank} over \emph{time}, but not
the actual amount in \faIcon{piggy-bank}. (We just deduce the ``inherent growth
rates'' \emph{based on} the observed amounts in \faIcon{piggy-bank}, and
calculate TWRR using those rates.)
\item \label{it:twrr-fmla}
In general, suppose there are \(m-1\) time points \(t_1,\dotsc,t_{m-1}\) where
the net contribution is nonzero (we shall denote them by \(C_1,\dotsc,C_{m-1}\)
respectively). Then, using similar notations as above\footnote{For example,
\(B_1,\dotsc,B_{m-1}\) are amounts/balances in \faIcon{piggy-bank} ``\emph{just
before}'' time \(t_1,\dotsc,t_{m-1}\) respectively.}, we have
\begin{align*}
B_0(1+j_1)&=B_1 \\
(B_1+C_1)(1+j_2)&=B_2 \\
\vdots\\
(B_{m-1}+C_{m-1})(1+j_m)&=B_m,
\end{align*}
and the \defn{time-weighted rate of return} (TWRR), sometimes denoted by
\(i^{\text{TW}}\),  is
\[
i^{\text{TW}}=\boxed{(1+j_1)\dotsb(1+j_m)-1}.
\]
\end{enumerate}
