\section{The Measurement of Interest}
\label{sect:interest-measurement}
\subsection{Origin of Interest}
\begin{enumerate}
\item We start with two people: \faIcon{user} and \faIcon{walking}. Suppose
\faIcon{user} lends his \faIcon{apple-alt} to \faIcon{walking}.
\item Terminologies:
\begin{itemize}
\item \faIcon{user}: \defn{lender}
\item \faIcon{walking}: \defn{borrower}
\item \faIcon{apple-alt}: \defn{capital} (owned by lender \faIcon{user})
\end{itemize}

\item In order to provide \emph{incentive} for the lender \faIcon{user} to lend
his \faIcon{apple-alt} to \faIcon{walking}, the borrower \faIcon{walking} needs
to give something to \faIcon{user}, which is known as \defn{interest}
\faIcon{arrow-right} main thing to be studied in this notes!

\item Usually, both capital and interest are expressed in terms of \emph{money}
\faIcon{dollar-sign}, which makes the calculations convenient.
\end{enumerate}
\subsection{Terminologies for Investment}
\begin{enumerate}
\item Here \emph{investment} is in the broad sense of putting aside
(\emph{invest}/\emph{lend}) a sum of money \faIcon{coins} (\emph{initial
investment}) into an ``investment fund'' \faIcon{piggy-bank} (playing the role
of ``borrower'') to get more money \faIcon{dollar-sign} (\emph{interest}) in
return (amount of \faIcon{dollar-sign} in \faIcon{piggy-bank} grows over time).

\begin{note}
The potential of earning more money in return by putting aside money for some
time is sometimes called \defn{time value of money}.
\end{note}

\item For the investment fund \faIcon{piggy-bank}, we shall impose the following assumptions:
\begin{itemize}
\item There is neither \emph{contribution} nor \emph{withdrawal} after the
initial investment, so that changes in ``value'' of (amount of money in) the
fund \faIcon{piggy-bank} are only due to \emph{interest}. \begin{note}
We will relax this assumption in \cref{sect:dcf-analysis}.
\end{note}
\item The ``growth rate'' for money in the fund \faIcon{piggy-bank} is constant
over time (regardless of the amount in the fund) and is independent from the
amount and timing of initial investment.
\end{itemize}

\item \defn{Principal} is the amount of initial investment made now (at time 0).
\item \defn{Accumulated value} (AV) at time \(t\) is the total amount
(principal + extra money in return) received when we withdraw all money from
the fund \faIcon{piggy-bank} at time \(t\) .

\begin{remark}
\item Accumulated value at time 0 = principal, since we simply receive back our
initial investment if we withdraw immediately from \faIcon{piggy-bank} at time
0 (there is no opportunity for ``growth''!).
\item Sometimes we just simply call ``accumulated value at time \(t\)'' as ``value at
time \(t\)''. (There is another kind of value; See \cref{subsect:present-val}.)
\end{remark}

\item \defn{Amount of interest} (between now and time \(t\)) is the extra money
in return (when withdrawing from \faIcon{piggy-bank} at time \(t\)), which is \(\text{time-\(t\) accumulated value}-\text{principal}\).

\item \defn{Measurement period} is the time \emph{unit} for ``time-related''
phrase like ``time \(t\)'', ``\(t\)th period'' etc.

\begin{note}
Unless otherwise specified, we assume the measurement period is \emph{years}
here \faIcon{arrow-right} e.g., ``1 period'' = ``1 year''.
\end{note}

\item \defn{Accumulation function} (denoted by \(a(t)\)) gives the
accumulated value of principal of 1 at time \(t\), for any \(t\ge 0\).

\begin{note}
Special case: \(a(0)=\text{principal}=1\).
\end{note}

\item \defn{Amount function} (denoted by \(A(t)\)) gives the
accumulated value of principal of \(k\) at time \(t\), for any \(t\ge 0\).

\begin{remark}
\item Special case: \(A(0)=\text{principal}=k\).
\item We have \(A(t)=ka(t)\) for any \(t\ge 0\) as the ``growth rate'' is
independent from the amount of initial investment.
\end{remark}

\item \defn{\(n\)th period} is the time period from time \(n-1\) to time \(n\),
for any \(n\in\N\).

\begin{warning}
It is \underline{not} the time period from time \(n\) to time \(n+1\)!
\end{warning}

Furthermore, time \(n\) (\(n-1\)) is called the \defn{end of \(n\)th period}
(\defn{beginning of \(n\)th period})

\begin{center}
\begin{tikzpicture}
\draw[-Latex] (0,0) -- (10,0) node[right]{Time};
\fill[] (0,0) circle [radius=0.05]
node[below] {0};
\fill[] (2,0) circle [radius=0.05]
node[below] {1};
\fill[] (4,0) circle [radius=0.05]
node[below] {2};
\fill[] (6,0) circle [radius=0.05]
node[below] {3};
\fill[] (8,0) circle [radius=0.05]
node[below] {4};
\draw[very thick, decorate,decoration={calligraphic brace, amplitude=5pt, raise=15pt, mirror}] (0,0) -- (2,0)
node[midway, below=0.7cm, font=\small]{1st period};
\draw[very thick, decorate,decoration={calligraphic brace, amplitude=5pt, raise=15pt, mirror}] (2,0) -- (4,0)
node[midway, below=0.7cm, font=\small]{2nd period};
\draw[very thick, decorate,decoration={calligraphic brace, amplitude=5pt, raise=15pt, mirror}] (4,0) -- (6,0)
node[midway, below=0.7cm, font=\small]{3rd period};
\draw[very thick, decorate,decoration={calligraphic brace, amplitude=5pt, raise=15pt, mirror}] (6,0) -- (8,0)
node[midway, below=0.7cm, font=\small]{4th period};
\draw[-Latex] (3.5,1) -- (4,0.2)
node[pos=-0.3]{beg.\ of 3rd period}
node[pos=-0.8]{end of 2nd period};
\draw[-Latex] (6.5,1) -- (6,0.2)
node[pos=-0.3]{end of 3rd period}
node[pos=-0.8]{beg.\ of 4th period};
\draw[-Latex] (0,1) -- (0,0.2)
node[pos=-0.3]{now};
\end{tikzpicture}
\end{center}

\item The \defn{amount of interest earned during \(n\)th period} (denoted by
\(I_n\)) is the increase in accumulated value during \(n\)th period
(accumulated value at the end of \(n\)th period subtracted by that at the
beginning of \(n\)th period):
\[
I_n=A(n)-A(n-1).
\]

\item Time-\(t\) accumulated value for an initial investment made at time
\(s>0\) (where \(t\ge s\)) is \(\displaystyle k\cdot\frac{a(t)}{a(s)}\).

\begin{center}
\begin{tikzpicture}
\draw[-Latex] (0,0) -- (10,0) node[right]{Time};
\fill[] (0,0) circle [radius=0.05]
node[below] {0}
node[above, brown]{1}
node[above=0.6cm, brown]{\faIcon{piggy-bank}};
\fill[] (4,0) circle [radius=0.05]
node[below=0.1cm] {\(s\)}
node[above, brown]{\(a(s)\)}
node[above=0.6cm, brown]{\faIcon{piggy-bank}}
node[below=0.6cm, violet]{\(k\)}
node[below=1.2cm, violet]{\faIcon{piggy-bank}};
\fill[] (8,0) circle [radius=0.05]
node[below=0.1cm] {\(t\)}
node[above, brown]{\(a(t)\)}
node[above=0.6cm, brown]{\faIcon{piggy-bank}}
node[below=0.5cm, violet]{\(k\cdot\frac{a(t)}{a(s)}\)}
node[below=1.2cm, violet]{\faIcon{piggy-bank}};
\draw[-Latex, brown] (5,0.6) -- (7,0.6);
\draw[-Latex, violet] (5,-1.2) -- (7,-1.2);
\draw[<->, magenta] (6,-1.1) -- (6,0.5)
node[pos=0.4, right, text width=1cm]{same growth};
\end{tikzpicture}
\end{center}
\end{enumerate}
\subsection{Interest Rates}
\label{subsect:interest-rates}
\begin{enumerate}
\item Interest rate describes the amounts of interest \faIcon{dollar-sign}
earned during different periods, and there are different kinds of interest rates.

\item \defn{Effective interest rate for \(n\)th period} (denoted by
\(i_n\)) is the ratio of interest earned during \(n\)th period to
the accumulated value at the \emph{beginning} of \(n\)th period, i.e.
\(I_n/A(n-1)\).
\item \label{it:effective-int-fmlas}
The effective interest rate \(i_n\) can be expressed in the following forms:
\begin{itemize}
\item \(\displaystyle i_n=\frac{A(n)-A(n-1)}{A(n-1)}\) or \(A(n)=A(n-1)(1+i_n)\) (in terms of \(A(\cdot)\)'s)
\item \(\displaystyle i_n=\frac{a(n)-a(n-1)}{a(n-1)}\) or \(a(n)=a(n-1)(1+i_n)\) (in terms of \(a(\cdot)\)'s)
\end{itemize}
\begin{center}
\begin{tikzpicture}
\draw[-Latex] (0,0) -- (10,0) node[right]{Time};
\fill[] (0,0) circle [radius=0.05]
node[below] {0};
\fill[] (2,0) circle [radius=0.05]
node[below] {1};
\fill[] (4,0) circle [radius=0.05]
node[below] {2};
\fill[] (6,0) circle [radius=0.05]
node[below] {3};
\fill[] (8,0) circle [radius=0.05]
node[below] {4};
\node[] (a4) at (8,0.3) {\(A(4)\)};
\node[] (a3) at (6,0.3) {\(A(3)\)};
\node[] (a2) at (4,0.3) {\(A(2)\)};
\node[] (a1) at (2,0.3) {\(A(1)\)};
\node[] (a0) at (0,0.3) {\(A(0)\)};
\draw[-Latex, brown] (a3.north east) to[bend left] (a4.north west);
\draw[-Latex, brown] (a2.north east) to[bend left] (a3.north west);
\draw[-Latex, brown] (a1.north east) to[bend left] (a2.north west);
\draw[-Latex, brown] (a0.north east) to[bend left] (a1.north west);
\node[brown] () at (8.5,0.9) {\(\cdots\)};
\node[brown] () at (7.2,1.2) {\(\times(1+i_4)\)};
\node[brown] () at (5.2,1.2) {\(\times(1+i_3)\)};
\node[brown] () at (3.2,1.2) {\(\times(1+i_2)\)};
\node[brown] () at (1.2,1.2) {\(\times(1+i_1)\)};
\end{tikzpicture}
\end{center}
\item For \emph{simple interest}, the amount of interest earned during each
period is a constant:
\[
\underbrace{\text{principal}}_{A(0)}\times\underbrace{\text{simple interest
rate}}_{i}.\]

\item Now fix a measurement period (time unit). Since a constant amount of
interest (\(A(0)i\)) is earned during each period, we can deduce the
accumulated value at the end of each period:
\begin{itemize}
\item \(A(1)=A(0)+A(0)i=A(0)(1+i)\)
\item \(A(2)=\underbrace{A(1)}_{\mathclap{A(0)(1+i)}}+A(0)i=A(0)(1+2i)\)
\item \(A(3)=\underbrace{A(2)}_{\mathclap{A(0)(1+2i)}}+A(0)i=A(0)(1+3i)\)
\item etc.
\end{itemize}
\item Generally, we have
\begin{itemize}
\item \(A(t)=A(0)(1+it)=k(1+it)\) for any \(t\in\N_0\)\footnote{\(\N_0=\{0,1,2,\dotsc\}.\)};
\item special case (\(k=1\)): \(a(t)=1+it\) for any \(t\in\N_0\).
\end{itemize}
\begin{center}
\begin{tikzpicture}
\begin{axis}[domain=0:5, ylabel=\(a(t)\), xlabel=\(t\), title={\(a(t)=1+it\) for any \(t\in\N_0\)}]
\addplot[violet, only marks, mark size=1pt, samples=6]{1+0.03*x};
\draw[pen colour=brown, very thick, decorate,decoration={calligraphic brace, amplitude=5pt, raise=5pt}] (1.1,1.03) -- (1.1,1)
node[brown, midway, right=10pt]{\(i\)};
\draw[pen colour=brown, very thick, decorate,decoration={calligraphic brace, amplitude=5pt, raise=5pt}] (2.1,1.06) -- (2.1,1.03)
node[brown, midway, right=10pt]{\(i\)};
\draw[pen colour=brown, very thick, decorate,decoration={calligraphic brace, amplitude=5pt, raise=5pt}] (3.1,1.09) -- (3.1,1.06)
node[brown, midway, right=10pt]{\(i\)};
\draw[pen colour=brown, very thick, decorate,decoration={calligraphic brace, amplitude=5pt, raise=5pt}] (4.1,1.12) -- (4.1,1.09)
node[brown, midway, right=10pt]{\(i\)};
\end{axis}
\end{tikzpicture}
\end{center}
\begin{remark}
\item The plot only contains discrete dots since only values at integer time points are
deduced.
\item When the measurement period chosen is shorter, the deduced values would
be more ``close in time'' (but they are still ``discrete'' in nature).
\end{remark}
\item We can observe that the dots in the plot are kind of ``linear'', so it
seems to be natural to join them by a straight line:
\begin{center}
\begin{tikzpicture}
\begin{axis}[domain=0:5, ylabel=\(a(t)\), xlabel=\(t\), title={\(a(t)=1+it\) for any \(t\ge 0\)}]
\addplot[blue]{1+0.03*x};
\addplot[violet, only marks, mark size=1pt, samples=6]{1+0.03*x};
\end{axis}
\end{tikzpicture}
\end{center}
From this we obtain the general definition of \defn{simple interest}:
\[
a(t)=1+it
\]
for any \(t\ge 0\), where \(i\) is the \defn{simple interest rate}
\faIcon{arrow-right} amount of interest earned during any time interval is
\emph{proportional} to the length of interval.

\item For \emph{compound interest}, the amount of interest earned during each
period is given by
\[
\text{accumulated value at the beginning of period}\times\underbrace{\text{compound interest rate}}_{i}.
\]

\item Similarly, we can deduce the accumulated value at the end of each period:
\begin{itemize}
\item \(A(1)=A(0)+A(0)i=A(0)(1+i)\)
\item \(A(2)=A(1)+A(1)i=\underbrace{A(1)}_{\mathclap{A(0)(1+i)}}(1+i)=A(0)(1+i)^{2}\)
\item \(A(3)=A(2)+A(2)i=\underbrace{A(2)}_{\mathclap{A(0)(1+i)^{2}}}(1+i)=A(0)(1+i)^{3}\)
\item etc.
\end{itemize}

\item Generally, we have
\begin{itemize}
\item \(A(t)=A(0)(1+i)^{t}=k(1+i)^{t}\) for any \(t\in\N_0\);
\item special case (\(k=1\)): \(a(t)=(1+i)^{t}\) for any \(t\in\N_0\).
\end{itemize}
\begin{center}
\begin{tikzpicture}
\begin{axis}[domain=0:5, ylabel=\(a(t)\), xlabel=\(t\), title={\(a(t)=(1+i)^t\) for any \(t\in\N_0\)}]
\addplot[violet, only marks, mark size=1pt, samples=6]{1.3^x};
\draw[pen colour=brown, very thick, decorate,decoration={calligraphic brace, amplitude=5pt, raise=5pt}] (1.1,1.3) -- (1.1,1)
;
\draw[pen colour=brown, very thick, decorate,decoration={calligraphic brace, amplitude=5pt, raise=5pt}] (2.1,\fpeval{1.3^2}) -- (2.1,1.3)
;
\draw[pen colour=brown, very thick, decorate,decoration={calligraphic brace, amplitude=5pt, raise=5pt}] (3.1,\fpeval{1.3^3}) -- (3.1,\fpeval{1.3^2})
;
\draw[pen colour=brown, very thick, decorate,decoration={calligraphic brace, amplitude=5pt, raise=5pt}] (4.1,\fpeval{1.3^4}) -- (4.1,\fpeval{1.3^3})
;
\end{axis}
\end{tikzpicture}
\end{center}

\item Since the dots in the plot are kind of ``exponential'', it is natural to
join them by an exponential function:
\begin{center}
\begin{tikzpicture}
\begin{axis}[domain=0:5, ylabel=\(a(t)\), xlabel=\(t\), title={\(a(t)=(1+i)^t\) for any \(t\ge 0\)}]
\addplot[violet, only marks, mark size=1pt, samples=6]{1.3^x};
\addplot[blue]{1.3^x};
\end{axis}
\end{tikzpicture}
\end{center}
This gives the general definition of \defn{compound interest}:
\[
a(t)=(1+i)^t
\]
for any \(t\ge 0\), where \(i\) is the \defn{compound interest rate}.  \item A
remarkable property for compound interest is that \emph{effective} interest
rate and \emph{compound} interest rate coincide: Under compound interest with
compound interest rate \(i\), the effective interest rate
\[
i_n=\frac{a(n)-a(n-1)}{a(n-1)}=\frac{(1+i)^{n}-(1+i)^{n-1}}{(1+i)^{n-1}}=1+i-1=i.
\]
for any \(n\in\N\).

\begin{remark}
\item Because of this ``nice'' property, compound interest is often used.
\item Here, unless otherwise specified, we shall assume compound interest, and
use the notation \(i\) to denote the constant (annual) compound interest rate
(which is also the effective interest rate for any year).
\end{remark}

\item In practice, interest (amount for whole period) is often only paid at the
end of each period \faIcon{arrow-right} ``jumps'' of amount of
\faIcon{dollar-sign} in \faIcon{piggy-bank} at discrete time points, rather
than in a ``continuous manner'' (interests paid/added to the fund
\faIcon{piggy-bank} simultaneously as they are earned \faIcon{arrow-right}
amount of \faIcon{dollar-sign} in \faIcon{piggy-bank} grows ``continuously'').
\begin{center}
\begin{tikzpicture}
\begin{axis}[domain=0:5, ylabel=\(a(t)\), xlabel=\(t\)]
\addplot[brown, const plot, mark=*, mark size=1pt, samples=6]{1.3^x};
\addplot[blue, opacity=0.3]{1.3^x};
\end{axis}
\end{tikzpicture}
\end{center}
\end{enumerate}
\subsection{Present Value}
\label{subsect:present-val}
\begin{enumerate}
\item \emph{Present value} captures the idea of ``current worth of future
money''. This concept is useful for answering questions like ``How much do we
need to invest in \faIcon{piggy-bank} to have an accumulated value of 1000 at
time 5?"
\item To find out the current worth, generally \emph{discounting} needs to be
performed, which can be understood as the ``reverse process'' of accumulating
(i.e., earning interest over time).
\begin{center}
\begin{tikzpicture}
\draw[-Latex] (0,0) -- (10,0) node[right]{Time};
\fill[] (0,0) circle [radius=0.05]
node[below] {0};
\fill[] (2,0) circle [radius=0.05]
node[below] {1};
\fill[] (4,0) circle [radius=0.05]
node[below] {2};
\fill[] (6,0) circle [radius=0.05]
node[below] {3};
\fill[] (8,0) circle [radius=0.05]
node[below] {4};
\node[] (ds) at (6,0.3) {\faIcon{dollar-sign}};
\node[] (cds) at (0,0.3) {\faIcon{question}};
\draw[-Latex, brown] (cds.north east) to[bend left] (ds.north west);
\node[brown] () at (3,1.7) {accumulate};
\draw[-Latex, violet] (ds.south west) to[bend left] (cds.south east);
\node[violet] () at (3,-1.3) {discount};
\draw[-Latex] (7.5,1.2) -- (6.3,0.5)
node[pos=-0.3]{future money};
\end{tikzpicture}
\end{center}
\item Given an effective interest rate \(i\), the \defn{discount factor}
(denoted by \(v\)) is given by \(v=1/(1+i)\).

\begin{note}
Multiplying the discount factor \(v\) = dividing by \(1+i\) = reverse process
of multiplying by \(1+i\) (accumulating for one period)
\faIcon{arrow-right} ``discount'' for one period.
\end{note}
\item Multiplying the discount factor \(v\) to an amount \faIcon{dollar-sign}
at the end of a period gives its \emph{value} (worth) at the beginning of the
period (i.e., the amount needed at that time in order to accumulate to
\faIcon{dollar-sign} at the end of the period).
\begin{center}
\begin{tikzpicture}
\draw[-Latex] (0,0) -- (10,0) node[right]{Time};
\fill[] (0,0) circle [radius=0.05]
node[below] {0};
\fill[] (2,0) circle [radius=0.05]
node[below] {1};
\fill[] (4,0) circle [radius=0.05] node[below] {2};
\fill[] (6,0) circle [radius=0.05]
node[below] {3};
\fill[] (8,0) circle [radius=0.05]
node[below] {4};
\node[] (ds) at (6,0.3) {\faIcon{dollar-sign}};
\node[] (cds) at (4,0.3) {value};
\draw[-Latex, brown] (cds.north east) to[bend left] (ds.north west);
\node[brown] () at (5.2,1.2) {\(\times(1+i)\)};
\draw[-Latex, violet] (ds.south west) to[bend left] (cds.south east);
\node[violet] () at (5.2,-0.5) {\(\times v\)};
\end{tikzpicture}
\end{center}
\item The \emph{time-0 value} of an amount \faIcon{dollar-sign} is known as its
\defn{present value} (PV).
\item More generally, we can discount for any number of periods. A ``more
general'' discount factor is given by the inverse function \(a^{-1}(t)\)
\faIcon{arrow-right} discounting for \(t\) periods: discount an amount
\faIcon{dollar-sign} at time \(t\) back to time 0.

\begin{note}
Still, this discounting serves as the reverse process of accumulating (for
\(t\) years). This applies similarly to other cases (discounting from a time
\(t\) to another time \(s<t\) \faIcon{arrows-alt-h} reverse process of
accumulating from time \(s\) to time \(t\) etc.)
\end{note}
\begin{center}
\begin{tikzpicture}
\draw[-Latex] (0,0) -- (10,0) node[right]{Time};
\fill[] (0,0) circle [radius=0.05]
node[below] {0}; \fill[] (2,0) circle [radius=0.05]
node[below] {1};
\fill[] (4,0) circle [radius=0.05]
node[below] {2};
\fill[] (6,0) circle [radius=0.05]
node[below] {3};
\fill[] (8,0) circle [radius=0.05]
node[below] {4};
\node[] (ds) at (6,0.3) {\faIcon{dollar-sign}};
\node[] (cds) at (0,0.3) {PV};
\draw[-Latex, brown] (cds.north east) to[bend left] (ds.north west);
\node[brown] () at (3,1.7) {\(\times a(t)\)};
\draw[-Latex, violet] (ds.south west) to[bend left] (cds.south east);
\node[violet] () at (3,-1.2) {\(\times a^{-1}(t)\)};
\end{tikzpicture}
\end{center}
\item The expressions for \(a^{-1}(t)\) under simple and compound interest are
respectively:
\begin{itemize}
\item simple interest: \(a^{-1}(t)=1/(1+it)\);
\item compound interest: \(a^{-1}(t)=(1+i)^{-t}=v^t\) (frequently used! \ystar)
\end{itemize}
\end{enumerate}
\subsection{Discount Rates}
\label{subsect:discount-rates}
\begin{enumerate}
\item \emph{Discount rate}, as its name suggests, is somewhat related to the
process of discounting.

\begin{warning}
Discount \emph{rate} and discount \emph{factor} are \underline{not}
the same!
\end{warning}
\item Discount rate may be seen as ``dual'' of interest rate, and we also have
three kinds of discount rates:
\begin{itemize}
\item effective discount rate \faIcon{arrows-alt-h} effective interest rate
\item simple discount rate \faIcon{arrows-alt-h} simple interest rate
\item compound discount rate \faIcon{arrows-alt-h} compound interest rate
\end{itemize}
\begin{note}
Nonetheless, there are some subtleties in developing discount rates that do not
present in the development of interest rates.
\end{note}

\item \defn{Effective discount rate for \(n\)th period} (denoted by
\(d_n\)) is the ratio of interest earned (or ``discount earned'') during
\(n\)th period to the accumulated value at the \emph{end} of \(n\)th
period, i.e.  \(I_n/A(n)\).

\item \label{it:effective-discount-fmlas}
The effective discount rate \(d_n\) can be expressed in the following forms:
\begin{itemize}
\item \(\displaystyle d_n=\frac{A(n)-A(n-1)}{A(n)}\) or \(A(n-1)=A(n)-\underbrace{A(n)d_n}_{\mathclap{\text{``discount earned''}}}=A(n)(1-d_n)\) (in terms of
\(A(\cdot)\)'s)
\item \(\displaystyle d_n=\frac{a(n)-a(n-1)}{a(n)}\) or \(a(n-1)=a(n)-\underbrace{a(n)d_n}_{\mathclap{\text{``discount earned''}}}=a(n)(1-d_n)\) (in terms of
\(a(\cdot)\)'s)
\end{itemize}
\begin{center}
\begin{tikzpicture}
\draw[-Latex] (0,0) -- (10,0) node[right]{Time};
\fill[] (0,0) circle [radius=0.05]
node[below] {0};
\fill[] (2,0) circle [radius=0.05]
node[below] {1};
\fill[] (4,0) circle [radius=0.05]
node[below] {2};
\fill[] (6,0) circle [radius=0.05]
node[below] {3};
\fill[] (8,0) circle [radius=0.05]
node[below] {4};
\node[] (a4) at (8,0.3) {\(A(4)\)};
\node[] (a3) at (6,0.3) {\(A(3)\)};
\node[] (a2) at (4,0.3) {\(A(2)\)};
\node[] (a1) at (2,0.3) {\(A(1)\)};
\node[] (a0) at (0,0.3) {\(A(0)\)};
\draw[-Latex, brown] (a4.north west) to[bend right] (a3.north east);
\draw[-Latex, brown] (a3.north west) to[bend right] (a2.north east);
\draw[-Latex, brown] (a2.north west) to[bend right] (a1.north east);
\draw[-Latex, brown] (a1.north west) to[bend right] (a0.north east);
\node[brown] () at (7,1.2) {\(\times(1-d_4)\)};
\node[brown] () at (5,1.2) {\(\times(1-d_3)\)};
\node[brown] () at (3,1.2) {\(\times(1-d_2)\)};
\node[brown] () at (1,1.2) {\(\times(1-d_1)\)};
\end{tikzpicture}
\end{center}
\begin{intuition}
About the term ``discount earned'': In each period the beginning value may be
seen as the ending value with ``discount'' (an amount subtracted). As time
passes, the ``discount'' is gradually ``earned back''.
\end{intuition}

\begin{remark}
\item From this, we can see that the effective discount rate is inherently ``backward''
\faIcon{arrow-right} ``discounting'' process.
\item An implicit requirement on \(d_n\) is that it must be less than 1 (for
\(A(n-1)\) to be positive \faIcon{arrow-right} ``sensible'').
\end{remark}
\item Of course one can reverse this ``reverse process'' and use the effective
discount rate \(d_n\) to ``move forward'':
\begin{itemize}
\item \(\displaystyle A(n)=A(n-1)\cdot\frac{1}{1-d_n}\)
\item \(\displaystyle a(n)=a(n-1)\cdot\frac{1}{1-d_n}\)
\end{itemize}
\begin{center}
\begin{tikzpicture}
\draw[-Latex] (0,0) -- (10,0) node[right]{Time};
\fill[] (0,0) circle [radius=0.05]
node[below] {0};
\fill[] (2,0) circle [radius=0.05]
node[below] {1};
\fill[] (4,0) circle [radius=0.05]
node[below] {2};
\fill[] (6,0) circle [radius=0.05]
node[below] {3};
\fill[] (8,0) circle [radius=0.05]
node[below] {4};
\node[] (a4) at (8,0.3) {\(A(4)\)};
\node[] (a3) at (6,0.3) {\(A(3)\)};
\node[] (a2) at (4,0.3) {\(A(2)\)};
\node[] (a1) at (2,0.3) {\(A(1)\)};
\node[] (a0) at (0,0.3) {\(A(0)\)};

\draw[-Latex, brown] (a4.north west) to[bend right] (a3.north east);
\draw[-Latex, brown] (a3.north west) to[bend right] (a2.north east);
\draw[-Latex, brown] (a2.north west) to[bend right] (a1.north east);
\draw[-Latex, brown] (a1.north west) to[bend right] (a0.north east);
\node[brown] () at (7,1.2) {\(\times(1-d_4)\)};
\node[brown] () at (5,1.2) {\(\times(1-d_3)\)};
\node[brown] () at (3,1.2) {\(\times(1-d_2)\)};
\node[brown] () at (1,1.2) {\(\times(1-d_1)\)};

\draw[-Latex, violet] (a3.south east) to[bend right] (a4.south west);
\draw[-Latex, violet] (a2.south east) to[bend right] (a3.south west);
\draw[-Latex, violet] (a1.south east) to[bend right] (a2.south west);
\draw[-Latex, violet] (a0.south east) to[bend right] (a1.south west);
\node[violet] () at (7,-0.8) {\(\displaystyle \times\frac{1}{1-d_4}\)};
\node[violet] () at (5,-0.8) {\(\displaystyle \times\frac{1}{1-d_3}\)};
\node[violet] () at (3,-0.8) {\(\displaystyle \times\frac{1}{1-d_2}\)};
\node[violet] () at (1,-0.8) {\(\displaystyle \times\frac{1}{1-d_1}\)};
\end{tikzpicture}
\end{center}
Compare this with the case for effective interest rate:
\begin{itemize}
\item \(\displaystyle A(n-1)=A(n)\cdot\frac{1}{1+i_n}\)
\item \(\displaystyle a(n-1)=a(n)\cdot\frac{1}{1+i_n}\)
\end{itemize}
\begin{center}
\begin{tikzpicture}
\draw[-Latex] (0,0) -- (10,0) node[right]{Time};
\fill[] (0,0) circle [radius=0.05]
node[below] {0};
\fill[] (2,0) circle [radius=0.05]
node[below] {1};
\fill[] (4,0) circle [radius=0.05]
node[below] {2};
\fill[] (6,0) circle [radius=0.05]
node[below] {3};
\fill[] (8,0) circle [radius=0.05]
node[below] {4};
\node[] (a4) at (8,0.3) {\(A(4)\)};
\node[] (a3) at (6,0.3) {\(A(3)\)};
\node[] (a2) at (4,0.3) {\(A(2)\)};
\node[] (a1) at (2,0.3) {\(A(1)\)};
\node[] (a0) at (0,0.3) {\(A(0)\)};

\draw[-Latex, brown] (a3.north east) to[bend left] (a4.north west);
\draw[-Latex, brown] (a2.north east) to[bend left] (a3.north west);
\draw[-Latex, brown] (a1.north east) to[bend left] (a2.north west);
\draw[-Latex, brown] (a0.north east) to[bend left] (a1.north west);
\node[brown] () at (7,1.2) {\(\times(1+i_4)\)};
\node[brown] () at (5,1.2) {\(\times(1+i_3)\)};
\node[brown] () at (3,1.2) {\(\times(1+i_2)\)};
\node[brown] () at (1,1.2) {\(\times(1+i_1)\)};

\draw[-Latex, violet] (a4.south west) to[bend left] (a3.south east);
\draw[-Latex, violet] (a3.south west) to[bend left] (a2.south east);
\draw[-Latex, violet] (a2.south west) to[bend left] (a1.south east);
\draw[-Latex, violet] (a1.south west) to[bend left] (a0.south east);
\node[violet] () at (7,-0.8) {\(\displaystyle \times\frac{1}{1+i_4}\)};
\node[violet] () at (5,-0.8) {\(\displaystyle \times\frac{1}{1+i_3}\)};
\node[violet] () at (3,-0.8) {\(\displaystyle \times\frac{1}{1+i_2}\)};
\node[violet] () at (1,-0.8) {\(\displaystyle \times\frac{1}{1+i_1}\)};
\end{tikzpicture}
\end{center}
\item For \emph{simple discount}, the amount of ``discount earned'' during each
period is constant.

\begin{warning}
The constant amount is \underline{not} ``\(A(0)i\)'' or ``\(A(0)d\)''.
\end{warning}

To determine the constant amount, first we need to fix an \(n\in \N\) as the
``last'' time point in our consideration. Then, the constant amount of
``discount earned'' in each period is given by
\[
A(n)\times\underbrace{\text{simple discount rate}}_{d}
\]
(where \(d<1\) to ensure positive accumulated values).
\item To deduce the accumulated values at different time points, we need to
work \emph{backward} from time \(n\) (``reverse process'' of the process for
the case of effective interest rate):
\begin{itemize}
\item \(A(n-1)=A(n)-A(n)d=A(n)(1-d)\)
\item \(A(n-2)=\underbrace{A(n-1)}_{A(n)(1-d)}-A(n)d=A(n)(1-2d)\)
\item \(\vdots\)
\item \(A(n-p)=A(n)\underbrace{(1-pd)}_{>0}\) {\color{ForestGreen}\faIcon{check}}
\item \(A(n-p-1)=A(n)\underbrace{(1-(p+1)d)}_{\le 0}\) {\color{red}\faIcon{times}}
\end{itemize}
\begin{note}
The equations are only valid (``sensible'') when RHS is positive, so we stop
this process once RHS becomes nonpositive.
\end{note}
\item The general form is thus \(A(n-k)=A(n)(1-kd)\) for any \(k=0,1,\dotsc,n\)
where RHS is positive (i.e., \(k<1/d\)).
\item \label{it:simple-discount-a-inv} Particularly, when \(n<1/d\), then we
have
\[
A(0)=A(n)\underbrace{(1-nd)}_{a^{-1}(n)}.
\]
\begin{center}
\begin{tikzpicture}
\draw[-Latex] (0,0) -- (10,0) node[right]{Time};
\fill[] (0,0) circle [radius=0.05]
node[below] {0}; \fill[] (2,0) circle [radius=0.05]
node[below] {1};
\fill[] (4,0) circle [radius=0.05]
node[below] {2};
\fill[] (6,0) circle [radius=0.05]
node[below] {3};
\fill[] (8,0) circle [radius=0.05]
node[below] {4};
\node[] (ds) at (6,0.3) {\(A(n)\)};
\node[] (cds) at (0,0.3) {\(A(0)\)};
\draw[-Latex, brown] (cds.north east) to[bend left] (ds.north west);
\node[brown] () at (3,1.7) {\(\times a(n)\)};
\draw[-Latex, violet] (ds.south west) to[bend left] (cds.south east);
\node[violet] () at (3,-1.2) {\(\times a^{-1}(n)\)};
\end{tikzpicture}
\end{center}

Consequently, as we fix different \(n\in\N\) satisfying \(n<1/d\), we have
\[
a^{-1}(n)=1-nd
\]
for any \(n\in\N\) with \(n<1/d\). (Of course the equation also holds for
\(n=0\) since \(a^{-1}(0)=1\) always.)
\begin{center}
\begin{tikzpicture}
\begin{axis}[domain=0:5, ylabel=\(a^{-1}(t)\), xlabel=\(t\), title={\(a^{-1}(t)=1-dt\) for any \(t\in\N_0\) with \(t<1/d\)}]
\addplot[violet, only marks, mark size=1pt, samples=6]{1-0.03*x};
\draw[pen colour=brown, very thick, decorate,decoration={calligraphic brace, amplitude=5pt, raise=5pt}] (1.1,1) -- (1.1,0.97)
node[brown, midway, right=10pt]{\(d\)};
\draw[pen colour=brown, very thick, decorate,decoration={calligraphic brace, amplitude=5pt, raise=5pt}] (2.1,0.97) -- (2.1,0.94)
node[brown, midway, right=10pt]{\(d\)};
\draw[pen colour=brown, very thick, decorate,decoration={calligraphic brace, amplitude=5pt, raise=5pt}] (3.1,0.94) -- (3.1,0.91)
node[brown, midway, right=10pt]{\(d\)};
\draw[pen colour=brown, very thick, decorate,decoration={calligraphic brace, amplitude=5pt, raise=5pt}] (4.1,0.91) -- (4.1,0.88)
node[brown, midway, right=10pt]{\(d\)};
\end{axis}
\end{tikzpicture}
\end{center}
\item It is again natural to join the dots by a straight line:
\begin{center}
\begin{tikzpicture}
\begin{axis}[domain=0:5, ylabel=\(a^{-1}(t)\), xlabel=\(t\), title={\(a^{-1}(t)=1-dt\) for any \(0\le t<1/d\)}]
\addplot[violet, only marks, mark size=1pt, samples=6]{1-0.03*x};
\addplot[blue]{1-0.03*x};
\end{axis}
\end{tikzpicture}
\end{center}
This gives the general definition of \defn{simple discount}:
\[
a^{-1}(t)=1-dt
\]
for any \(0\le t<1/d\), where \(d\) is the \defn{simple discount rate}.

\begin{warning}
It is \underline{not} \(a(t)=1-dt\) in the definition! We consider the \emph{inverse
function} \(a^{-1}(t)\) instead.  Alternatively, one may express the definition
as
\[
a(t)=\frac{1}{1-dt}
\]
for any \(0\le t<1/d\) \faIcon{arrow-right} \(a(t)\) is \emph{nonlinear}!
\end{warning}
\item For \emph{compound discount}, the amount of ``discount earned'' each
period is
\[
\text{accumulated value at the \emph{end} of period}\times\underbrace{\text{compound discount rate}}_{d}
\]
(where \(d<1\)).
\item Likewise, we work backward from a fixed time \(n\):
\begin{itemize}
\item \(A(n-1)=A(n)-A(n)d=A(n)(1-d)\)
\item \(A(n-2)=A(n-1)-A(n-1)d=\underbrace{A(n-1)}_{A(n)(1-d)}(1-d)=A(n)(1-d)^{2}\)
\item \(\vdots\)
\item \(A(0)=A(n)(1-d)^{n}\)
\end{itemize}
(RHS is always positive since \(1-d>0\).)
\item Using a similar argument as \labelcref{it:simple-discount-a-inv}, we have
\[
a^{-1}(t)=(1-d)^{t}
\]
for any \(t\in\N_0\).

\begin{center}
\begin{tikzpicture}
\begin{axis}[domain=0:5, ylabel=\(a^{-1}(t)\), xlabel=\(t\), title={\(a^{-1}(t)=(1-d)^{t}\) for any \(t\in\N_0\)}]
\addplot[violet, only marks, mark size=1pt, samples=6]{(1-0.3)^x};
\draw[pen colour=brown, very thick, decorate,decoration={calligraphic brace, amplitude=5pt, raise=5pt}] (1.1,1) -- (1.1,0.7)
;
\draw[pen colour=brown, very thick, decorate,decoration={calligraphic brace, amplitude=5pt, raise=5pt}] (2.1,0.7) -- (2.1,\fpeval{0.7^2})
;
\draw[pen colour=brown, very thick, decorate,decoration={calligraphic brace, amplitude=5pt, raise=5pt}] (3.1,\fpeval{0.7^2}) -- (3.1,\fpeval{0.7^3})
;
\draw[pen colour=brown, very thick, decorate,decoration={calligraphic brace, amplitude=5pt, raise=5pt}] (4.1,\fpeval{0.7^3}) -- (4.1,\fpeval{0.7^4})
;
\end{axis}
\end{tikzpicture}
\end{center}
\item Like the case for compound interest rate, it is natural to join the dots
by an exponential function:
\begin{center}
\begin{tikzpicture}
\begin{axis}[domain=0:5, ylabel=\(a^{-1}(t)\), xlabel=\(t\), title={\(a^{-1}(t)=(1-d)^{t}\) for any \(t\ge 0\)}]
\addplot[violet, only marks, mark size=1pt, samples=6]{(1-0.3)^x};
\addplot[blue]{(1-0.3)^x};
\end{axis}
\end{tikzpicture}
\end{center}
This leads to the general definition of \defn{compound discount}:
\[
a^{-1}(t)=(1-d)^{t}
\]
for any \(t\ge 0\).

\begin{remark}
\item Alternatively, we can express the definition as
\[
a(t)=\frac{1}{(1-d)^{t}}
\]
for any \(t\ge 0\).
\item Like interest rate, unless otherwise specified, we shall assume \emph{compound
discount} and use the notation \(d\) to denote the (annual) compound discount
rate.
\end{remark}
\end{enumerate}
\subsection{Relationship Between Interest and Discount Rates}
\begin{enumerate}
\item We have separately discussed interest and discount rates in
\cref{subsect:interest-rates,subsect:discount-rates}. So naturally, the next
question to ask is how are they \emph{related}.
\item An important concept that connects two rates is the concept of
\emph{equivalency}.
\item Two interest/discount rates are \defn{equivalent} if a fixed principal
\faIcon{dollar-sign} invested at each of the rates gives the \emph{same
accumulated value} at any fixed time point.

\begin{note}
More explicitly, let \(r_1\) and \(r_2\) be the two rates. Then it means
\[
a(t)\;@\;r_1 = a(t)\;@\;r_2
\]
for any \(t\ge 0\).
\end{note}

\item Example: For compound interest rate \(i\) and compound discount rate
\(d\), they are equivalent when
\[
(1+i)^{t}=\frac{1}{(1-d)^{t}}\quad\text{for any \(t\ge 0\)},
\]
which implies \(\displaystyle 1+i=\frac{1}{1-d}\).

\begin{note}
Unless otherwise specified, we shall assume the rates \(i\) and \(d\) are
equivalent (when we use the notations ``\(i\)'' and ``\(d\)'').
\end{note}

\item \label{it:id-equiv-fmlas}
The following are some results under the equivalency between \(i\) and \(d\):
\begin{itemize}
\item \(\displaystyle i=\frac{d}{1-d}\)
\item \(\displaystyle d=\frac{i}{1+i}\)
\item \(v=1-d\)
\end{itemize}
(All of them follow readily from rearranging the equation \(1+i=1/(1-d)\).)

\item We can observe from \cref{subsect:discount-rates} that the formulas for
discount rates appear to be ``more messy'', hence the equivalency between \(i\)
and \(d\) should be the main (but not only!) tool you are using for
calculations involving discount rates.
\end{enumerate}
\subsection{Nominal Interest and Discount Rates}
\begin{enumerate}
\item \emph{Nominal rate}, as suggested by its name, is only nominal (``in name
only'') and is not an \emph{actual} rate that is directly used in the
calculations. It serves as a \emph{reference} for computing \emph{actual}
interests.
\item Example: When a bank \faIcon{landmark} offers 12\% loan interest rate
(per annum) ``compounded monthly'', ``12\%'' serves as a reference only: It
does not simply mean the annual effective interest rate is 12\%. Rather, it
means
\begin{itemize}
\item the measurement period is months (signified by ``monthly''), while the
``base'' time unit is years (signified by ``per annum'');
\item compound interest is assumed (signified by ``compounded'');
\item the compound interest rate (for each month) is
\(12\%{\color{violet}/12}=1\%\). (Number of months in a period in ``base'' time
unit = 12.)
\end{itemize}
\begin{note}
Unless otherwise specified, we assume that the ``base'' time unit is years for
any nominal rate.
\end{note}
\item More generally, we can define nominal interest rate as follows. For the
case of \defn{compounded \(m\)thly} with the (annual) \defn{nominal interest
rate compounded \(m\)thly} (denoted by \(i^{(m)}\)),
\begin{itemize}
\item the measurement period is \(1/m\) of years;
\item compound interest is assumed;
\item the compound interest rate is \(i^{(m)}/m\).
\end{itemize}
In this case, the accumulation function is given by
\[
a(s)=\qty(1+\frac{i^{(m)}}{m})^{s}
\]
for any \(s\ge 0\), \emph{where the time unit of \(s\) is \(1/m\) of years}, or

\[
a(t)=\qty(1+\frac{i^{(m)}}{m})^{mt}
\]
for any \(t\ge 0\), \emph{where the time unit of \(t\) is years}.

\begin{note}
Time \(t\) in years is time \(mt\) in \(1/m\) of years.
\begin{center}
\begin{tikzpicture}
\draw[-Latex] (0,0) -- (10,0) node[right]{Time (in years)};
\foreach \x in {0,...,4}
\fill[] (2*\x,0) circle [radius=0.05]
node[below] {\x};

\draw[-Latex] (0,-1) -- (10,-1) node[right]{Time (in half years)};
\foreach \x in {0,...,8}
\fill[] (\x,-1) circle [radius=0.05]
node[below] {\x};

\draw[-Latex] (0,-2) -- (10,-2) node[right]{Time (in quarters)};
\foreach \x in {0,...,16}
\fill[] (0.5*\x,-2) circle [radius=0.05]
node[below] {\x};
\end{tikzpicture}
\end{center}
\end{note}

\begin{warning}
Make sure that you are clear about the time unit being used!
\end{warning}
\item For the case of \emph{compounded \(m\)thly} with the (annual)
\defn{nominal discount rate compounded \(m\)thly} (denoted by \(d^{(m)}\)),
\begin{itemize}
\item the measurement period is \(1/m\) of years;
\item compound interest is assumed;
\item the compound discount rate is \(d^{(m)}/m\).
\end{itemize}
In this case, the accumulation function is
\[
a(s)=\qty(1-\frac{d^{(m)}}{m})^{{\color{magenta}-}s}
\]
for any \(s\ge 0\), \emph{where the time unit of \(s\) is \(1/m\) of years}, or

\[
a(t)=\qty(1-\frac{d^{(m)}}{m})^{{\color{magenta}-}mt}
\]
for any \(t\ge 0\), \emph{where the time unit of \(t\) is years}.
\item \label{it:nominal-effective-equiv-rates-fmlas}
We can also talk about the concept of equivalency for nominal rates:
\begin{itemize}
\item when compound effective interest rate \(i\) and nominal interest rate
\(i^{(m)}\) are equivalent,
\[
(1+i)^{t}=\qty(1+\frac{i^{(m)}}{m})^{mt}\;\forall t\ge 0
\implies
\boxed{1+i=\qty(1+\frac{i^{(m)}}{m})^{m}}.
\]
\begin{warning}
We need to use the same time unit for both sides!
\end{warning}

\item when compound effective discount rate \(d\) and nominal interest rate
\(d^{(m)}\) are equivalent,
\[
(1-d)^{-t}=\qty(1-\frac{d^{(m)}}{m})^{-mt}\;\forall t\ge 0
\implies
\boxed{1-d=\qty(1-\frac{d^{(m)}}{m})^{m}}.
\]
\end{itemize}
\end{enumerate}

\subsection{Force of Interest and Discount}
\begin{enumerate}
\item Loosely, (annual) force of interest (discount) is the ``nominal rate''
``\(i^{(\infty)}\)'' (``\(d^{(\infty)}\)'' resp.):
\begin{center}
\begin{tikzpicture}
\node[] () at (0,0) {compounded \(1/m\)thly};
\node[] () at (10,0) {compounded ``continuously''};
\draw[-Latex] (3,0) -- (7,0)
node[midway, above]{\(m\to\infty\)};
\node[] () at (0,-1) {\(i^{(m)}\) (\(d^{(m)}\))};
\node[] () at (10,-1) {\(i^{(\infty)}\) (\(d^{(\infty)}\))};
\draw[<->, magenta] (1,-1) -- (9,-1)
node[midway, below]{same numerical value};
\end{tikzpicture}
\end{center}
\item More precisely, we fix the nominal interest and discount rates compounded
\(m\)thly at some values \(i^{(\infty)}\) and \(d^{(\infty)}\) respectively.
Then, consider the limits of the accumulation function \(a(t)\) for the
compounded \(m\)thly case at each of the rates as \(m\to\infty\): For any
\(t\ge 0\),
\[
\lim_{m\to \infty}\qty(1+\frac{i^{(\infty)}}{m})^{mt}=e^{i^{(\infty)}t},
\]
and
\[
\lim_{m\to \infty}\qty(1-\frac{d^{(\infty)}}{m})^{-mt}=e^{-d^{(\infty)}(-t)}
=e^{d^{(\infty)}t}.
\]
We shall fix the values \(i^{(\infty)}\) and \(d^{(\infty)}\) such that they
are equivalent asymptotically (``at limit''), i.e.,
\[
e^{i^{(\infty)}t}=e^{d^{(\infty)}t}\quad\text{for any \(t\ge 0\)},
\]
which implies \(i^{(\infty)}=d^{(\infty)}\).

\item \label{it:foi-fod}
We denote this common value \(i^{(\infty)}=d^{(\infty)}\) by \(\delta\)
and call it \defn{force of interest} (or \defn{force of discount}).

For the \defn{compounded continuously} case with (annual) force of interest
\(\delta\) (or interest rate \(\delta\) \emph{compounded continuously}), the
accumulation function is given by
\[
a(t)=e^{\delta t}\quad\text{for any \(t\ge 0\)}.
\]
\item \label{it:foi-effective-equiv-fmla}
Then, when the effective interest rate \(i\) and force of interest \(\delta\)
are equivalent, we have
\[
(1+i)^t=e^{\delta t}\;\forall t\ge 0
\implies \boxed{\delta=\ln(1+i)}.
\]
\item \labelcref{it:foi-fod} gives the definition of \emph{constant} force of
interest. Generally, in the compounded continuously case, we can indeed allow
\emph{varying} force of interest , whose definition can be motivated through an
``infinitesimal'' argument (see \labelcref{it:varying-foi-motivation}).

\item \label{it:varying-foi-motivation}
For the compounded continuously case with constant force of interest
\(\delta\), we have for any \(t\ge 0\), \(a(t)=e^{\delta t}\), which implies
\(A(t)=A(0)e^{\delta t}\). Hence, the interest earned during the time interval
\([t, t+h]\) is
\[
A(t+h)-A(t)=\underbrace{A(t)}_{A(0)e^{\delta t}}(e^{\delta h}-1)
\approx A(t)(1+\delta h-1)=A(t)\delta h.
\]
(The approximation works better when \(h\) is smaller.)
After that, loosely speaking, the interest earned in the infinitesimal time
interval \([t,t+\dd{t}]\) is \(A(t)\delta\dd{t}\).

Now, for the case with \emph{varying} force of interest \(\delta_t\) (as a
function of time \(t\)), the interest earned in \([t,t+\dd{t}]\) becomes
\(A(t)\delta_t\dd{t}\).\footnote{Loosely speaking, during the infinitesimal time
interval \([t,t+\dd{t}]\), the force of interest can be ``regarded'' as staying at
its time-\(t\) value \(\delta_t\) since the time length involved is
``infinitesimally small'' \faIcon{arrow-right} ``constant'' force of interest
during the interval.} Thus, we have
\[
\dd{A(t)}=\underbrace{A(t+\dd{t})-A(t)}_{\mathclap{\text{interest earned in \([t,t+\dd{t}]\)}}}=A(t)\delta_t\dd{t}
\implies
\dv{A(t)}{t}=A(t)\delta_t.
\]
It follows that
\[
\delta_t=\frac{A'(t)}{A(t)}
\]
which gives the definition of \defn{force of interest at time \(t\)}.

\item From this definition of varying force of interest, we can derive the
following result.
\begin{proposition}
\label{prp:foi-fmlas}
For any time \(t\),
\begin{enumerate}
\item \(\displaystyle a(t)=\exp(\int_{0}^{t}\delta_s\dd{s})\);
\item \(\displaystyle A(t)=A(0)+\int_{0}^{t}A(s)\delta_s\dd{s}\).
\end{enumerate}
\end{proposition}
\begin{intuition}
\begin{itemize}
\item For the first formula, firstly for any time \(s\) we have \(A(s+h)\approx
A(s)e^{\delta_sh}\) when \(h\) is small. Hence,
\[
a(t)=\frac{A(t)}{A(0)}=\frac{A(h)}{A(0)}\frac{A(2h)}{A(h)}\dotsb\frac{A(t)}{A(t-h)}
\approx\exp(\delta_0h+\delta_1h+\dotsb+\delta_{t-h}h)
\]
Then, loosely, letting \(h\to 0\) gives
\[
a(t)=\exp(\int_{0}^{t}\delta_s\dd{s}).
\]
\item For the second formula, \(A(s)\delta_s\dd{s}\) can be regarded as
``interest earned in infinitesimal time interval \([s,s+\dd{s}]\)'', and thus
\(\displaystyle \int_{0}^{t}A(s)\delta_s\dd{s}\) can be interpreted as ``sum''
of all such interests \faIcon{arrow-right} amount of interest earned in time
interval \([0,t]\).
\end{itemize}
\end{intuition}

\begin{pf}
Firstly, by chain rule we have \(\displaystyle \delta_s=\dv{}{s}\ln A(s)\).
Thus,
\[
\int_{0}^{t}\delta_s\dd{s}=\ln A(t)-\ln A(0)=\ln(\frac{A(t)}{A(0)})=\ln a(t),
\]
and then the first formula follows.

Next, note that \(A(s)\delta_s=A'(s)\), so
\[
\int_{0}^{t}A(s)\delta_s\dd{s}=A(t)-A(0),
\]
proving the second formula.
\end{pf}
\end{enumerate}
