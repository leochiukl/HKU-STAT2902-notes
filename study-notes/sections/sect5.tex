\section{Bonds and Other Securities}
\label{sect:bonds-other-securities}
\begin{enumerate}
\item A \defn{security} is a tradable financial asset.
\item Here, we will discuss the following securities:
\begin{itemize}
\item bonds (main focus)
\item common stocks
\item preferred stocks
\end{itemize}
For other kinds of securities like forwards, futures, and options, see STAT3905.
\end{enumerate}
\subsection{Bonds}
\begin{enumerate}
\item A \defn{bond} is an interest-bearing security which promises to pay
stated amount(s) of money at some future date(s).

\begin{remark}
\item This definition is rather general, so there are indeed many kinds of
bonds. But here we shall focus on some ``simple'' bonds. (Nonetheless, we will
briefly discuss some variants in
\cref{subsect:serial-bonds,subsect:callable-bonds}.)
\item \defn{Coupons} \faIcon{money-bill-wave} are periodic \emph{level}
payments promised in a bond, which are made at the end of each period (year
unless otherwise specified).
\item A bond with nonzero (zero) coupons is called a \defn{coupon-paying bond}
(\defn{zero-coupon bond}).
\end{remark}
\item For simplicity, we shall make the
following assumptions here:
\begin{enumerate}
\item all obligations for paying money are met --- there is no \emph{default}
(``breaking promise'') \begin{note}
See STAT3904 for a way to handle the \emph{default risk}.
\end{note};
\item bond has a fixed \defn{maturity} (i.e., time at which all promised
payments are made)
\begin{note}
This will be loosened when we discuss
\emph{callable bonds} later in \cref{subsect:callable-bonds}.
\end{note};
\item we are only interested in finding bond price (given a ``yield
rate''\footnote{This is related to the concept of IRR in
\cref{sect:dcf-analysis}.}) \emph{immediately after} a ``coupon'' payment date,
or at time 0 \begin{note} Here time 0 is the \emph{purchasing time} of the
bond.
\end{note}.
\end{enumerate}
\begin{note}
The \defn{term} of a bond is the time length between time 0 and the maturity date.
For a bond with \(n\) periods term, we call it \defn{\(n\)-period bond}.
\end{note}
\item \label{it:bond-main-qs}
The main questions of interest here are:
\begin{enumerate}
\item Given a specific yield rate, what is the bond price (at time 0, without
loss of generality\footnote{For time-\(k\) price of an \(n\)-year bond (where
``time \(k\)'' here is supposed to be the time immediately after payment of
\(k\)th coupon), it is the same as the time-0 price of an otherwise identical
bond, but with term \(n-k\) years.})?

\item Given a time-0 bond price (purchasing price), what is the (implied) yield rate?
\begin{note}
Loosely, this means ``How high is the return if the bond is purchased at this
price?''
\end{note}
\end{enumerate}
\item To answer these questions, it is useful to introduce some notations for
describing a bond:
\begin{itemize}
\item \(F\): \defn{face value}/\defn{par value}/\defn{nominal value} (value
``printed'' on the ``face'' of bond, used as reference for calculating payment
amount(s))
\item \(P\): purchasing price (time-0 price) of the bond
\item \(A\): purchasing price per unit nominal (i.e., expressed as a unit where 1 unit = \(F\))

\begin{note}
\(A\) units = \(P\) by definition, so \(A=P/F\).
\end{note}
\item \(r\): \defn{coupon rate} (or \defn{nominal yield}), which is the
\emph{total} amount of coupon(s) in each year expressed as a fraction of face value:
\(\displaystyle r=\frac{\text{annual coupon amount}}{F}\)
\item \(n\): total number of coupons to be paid throughout
\item \(C\): redemption value, i.e., amount of payment promised to be made at
\emph{maturity} (\warn{} \underline{not} including coupon at that time!)
\item \(g\): \defn{modified coupon rate}, i.e., annual coupon amount expressed
as a fraction of \emph{redemption value}: \(\displaystyle g=\frac{\text{annual coupon amount}}{C}\)
\item \(R\): redemption value per unit nominal: \(R=C/F\); the bond is
\begin{itemize}
\item \defn{redeemable at par} if \(R=1\) (redemption value = par value);
\item \defn{redeemable above par} if \(R>1\) (redemption value \(>\) par value);
\item \defn{redeemable below par} if \(R<1\) (redemption value \(<\) par value)
\end{itemize}
\item \(i\): \defn{yield rate}/\defn{yield to maturity}/\defn{yield to
redemption} --- By treating initial payment of purchasing price \(P\)  as cash
outflow and the rest of incomes from bond as cash inflows, we can regard the
bond as a ``project'', and then \(i\) is simply the yield rate (IRR) of the
``project''

\begin{note}
The terms ``to maturity'' and ``to redemption'' are related to the
interpretation of IRR: rate of return for a ``fund'' \faIcon{piggy-bank} which
accumulates \emph{to maturity/redemption}.
\end{note}
\item \(K\): present value of \(C\) (treated as an amount at \emph{maturity})
at yield rate \(i\)
\item \(G\): \defn{base amount}, i.e., the amount we need to invest into a fund
\faIcon{piggy-bank} at rate \(i\) such that the periodic interest payments
(assumed to be paid at the end of each period) are identical to the coupon
payments from the bond \faIcon{arrow-right} \(Gi=Fr\).
\end{itemize}
\begin{center}
\begin{tikzpicture}
\draw[-Latex] (0,0) -- (12,0);
\node[draw] () at (6,3.5) {Graphical Illustration};
\draw[fill] (0,0) circle [radius=0.5mm];
\draw[fill] (2,0) circle [radius=0.5mm];
\draw[fill] (4,0) circle [radius=0.5mm];
\draw[fill] (8,0) circle [radius=0.5mm];
\draw[fill] (10,0) circle [radius=0.5mm];
\node[] () at (0,-0.5) {0};
\node[] () at (2,-0.5) {1};
\node[] () at (4,-0.5) {2};
\node[] () at (6,-0.5) {\(\cdots\)};
\node[] () at (8,-0.5) {\(n-1\)};
\node[] () at (10,-0.5) {\(n\)};
\node[] () at (0,0.5) {\(-P\)};
\node[] () at (2,0.5) {\(+Fr\)};
\node[] () at (4,0.5) {\(+Fr\)};
\node[] () at (6,0.5) {\(\cdots\)};
\node[] () at (8,0.5) {\(+Fr\)};
\node[] () at (10,0.5) {\(+Fr+C\)};

\node[] () at (2,1) {or};
\node[] () at (4,1) {or};
\node[] () at (8,1) {or};
\node[] () at (10,1) {or};

\node[] () at (2,1.5) {\(+Cg\)};
\node[] () at (4,1.5) {\(+Cg\)};
\node[] () at (6,1.5) {\(\cdots\)};
\node[] () at (8,1.5) {\(+Cg\)};
\node[] () at (10,1.5) {\(+Cg+C\)};

\node[] () at (2,2) {or};
\node[] () at (4,2) {or};
\node[] () at (8,2) {or};
\node[] () at (10,2) {or};


\node[] () at (2,2.5) {\(+Gi\)};
\node[] () at (4,2.5) {\(+Gi\)};
\node[] () at (6,2.5) {\(\cdots\)};
\node[] () at (8,2.5) {\(+Gi\)};
\node[] () at (10,2.5) {\(+Gi+C\)};
\end{tikzpicture}
\end{center}
\item \label{it:bond-pricing-fmlas}
Now we are ready to answer the first question in
\labelcref{it:bond-main-qs}. To determine the time-0 bond price given a
specific yield rate \(i\), we have the following four formulas:
\begin{enumerate}
\item basic formula: \(P=Fr\ax{\angl{n}i}+Cv^n\)
\item premium/discount formula: \(P=C+C(g-i)\ax{\angl{n}i}\)
\item base amount formula: \(P=G+(C-G)v^n\)
\item Makeham's formula: \(\displaystyle P=K+\frac{g}{i}(C-K)\)
\end{enumerate}
\item \label{it:bond-basic-fmla}
For basic formula, it is given by
\[
P=Fr\ax{\angl{n}i}+\underbrace{Cv^n}_{K},
\]
which directly follows from the fact that \(i\) is the IRR of the ``project''
and the definition of IRR.

\item \label{it:bond-prem-disc-fmla}
To derive premium/discount formula, we start from the basic formula:
\[
P=Fr\ax{\angl{n}i}+Cv^n
=Fr\ax{\angl{n}i}+C(1-i\ax{\angl{n}i})
=C+(\underbrace{Fr}_{Cg}-Ci)\ax{\angl{n}i}
=\boxed{C+C(g-i)\ax{\angl{n}i}}.
\]
To understand why the formula is called ``premium/discount formula'', we first
introduce the following terminologies: The bond is
\begin{itemize}
\item \defn{sold at premium} if \(P>C\) (purchasing price is ``in excess of'' redemption value \faIcon{arrow-right} ``premium'');
\item \defn{sold at discount} if \(P<C\) (purchasing price is ``cheaper than'' redemption value \faIcon{arrow-right} ``discount'').
\end{itemize}
When the bond is sold at premium (discount), the \defn{premium}
(\defn{discount}) is given by \(P-C\) (\(C-P)\) (the difference between
purchasing price and redemption value).

Thus, the term \(C(g-i)\ax{\angl{n}i}\) in the premium/discount formula is the
premium or discount (in absolute value) \faIcon{arrow-right} hence
``premium/discount''.

\begin{note}
By premium/discount formula, we can see that the bond is sold at premium (discount) iff \(g>i\) (\(g<i\)).
\end{note}

\item \label{it:bond-base-amt-fmla}
For deriving base amount formula, consider:
\[
P=Fr\ax{\angl{n}i}+Cv^n
=G\underbrace{i\ax{\angl{n}i}}_{1-v^n}+Cv^n
=\boxed{G+(C-G)v^n}.
\]
The main feature of base amount formula is that there is not
``\(\ax{\angl{n}i}\)'' and we just have ``\(v^n\)'' \faIcon{arrow-right}
relatively simpler to compute.
\item \label{it:bond-makeham-fmla}
Finally, Makeham's formula can be derived as follows:
\[
P=Fr\ax{\angl{n}i}+Cv^n
=Cg\underbrace{\ax{\angl{n}i}}_{\mathclap{\frac{1-v^n}{i}}}+K
=\boxed{K+\frac{g}{i}(C-K)}.
\]
\begin{intuition}
By comparing with basic formula, we see that \(\displaystyle \frac{g}{i}(C-K)\)
is the PV of all coupons (\(Fr\ax{\angl{n}i}\)). Then, consider:
\begin{itemize}
\item \(g>i\) \faIcon{arrow-right} sold at premium \faIcon{arrow-right} PV of coupons is relatively ``higher'' as \(g/i>1\). (We ``pay extra for it'')
\item \(g<i\) \faIcon{arrow-right} sold at discount \faIcon{arrow-right} PV of coupons is relatively ``lower'' as \(g/i<1\). (We ``pay less, so have less''.)
\end{itemize}
\end{intuition}
\item Answering the second question in \labelcref{it:bond-main-qs} is quite
straightforward since essentially we are just finding IRR of a ``project'', so
standard approaches (e.g.\ numerical) apply.
\end{enumerate}
\subsection{Incorporating Income and Capital Gains Taxes}
\begin{enumerate}
\item In practice, there are taxes charged on the bond payments. Two typical
taxes are income and capital gain taxes. Here we consider the situation that
\emph{income} tax is payable for each coupon payment (treated as ``income''),
and \emph{capital gain} tax is payable (at the maturity) \emph{in case} the redemption
value (\(C\)) exceeds purchasing price (\(P\)) (the difference \(C-P\) is
treated as ``capital gain'').

\item \label{it:inc-tax-basic-prem-desc-makeham-fmlas}
First consider the case with income tax (and no capital gains tax). Let
\(t_{\text{inc}}\) be the income tax rate. Then, after charging income tax on
each coupon payment, the amount of each coupon decreases from \(Fr\) to
\(Fr(1-t_{\text{inc}})\). After making such modification on the coupon amount,
we can obtain the following bond pricing formulas:
\begin{enumerate}
\item basic formula: \(P=Fr(1-t_{\text{inc}})\ax{\angl{n}i}+Cv^n\)
\item premium/discount formula: \(P=C+C\qty[g(1-t_{\text{inc}})-i]\ax{\angl{n}i}\)
\item Makeham's formula: \(\displaystyle P=K+\frac{g(1-t_{\text{inc}})}{i}(C-K)\)
\end{enumerate}

\item \label{it:cap-gain-criterion}
To incorporate capital gains tax, it is more tricky since we also need to
decide \emph{whether} there is a capital gain (and hence capital gains tax). As
mentioned previously, there is a capital gain iff \(C>P\). Based on
premium/discount formula, we know that the condition is equivalent to:
\begin{itemize}
\item \(g<i\) (no income tax), or
\item \(g(1-t_{\text{inc}})<i\) (income tax rate: \(t_{\text{inc}}\)).
\end{itemize}
\begin{note}
This equivalent criterion is helpful for deciding whether there is capital
gains tax payable.
\end{note}

\item \label{it:handle-cap-gain-tax}
Now, suppose that capital gains tax is payable, and let \(t_{\text{cap}}\) be
the capital gains tax rate. Then, the capital gains tax payable at the maturity
is \((C-P)t_{\text{cap}}\), which can be treated as an extra cash outflow (from
the perspective of bond purchaser) at the maturity. Hence, we just need to add
\((C-P)t_{\text{cap}}v^n\) at the RHS of bond pricing formulas in
\labelcref{it:bond-pricing-fmlas} to handle the capital gains tax.
\end{enumerate}
\subsection{Incorporating Inflation}
\begin{enumerate}
\item Another practical element involved is \emph{inflation}, which affects the
\emph{real} amounts of future cash flows.

\item \label{it:real-pv-fmla}
To track the inflation, we usually utilize a \emph{price level index},
denoted by \(Q(t)\) for time-\(t\) value. The defining property of a
\defn{price level index} is that the \defn{real present value} of a time-\(t\) cash flow
\(Q(t)(1+i)^t\) is \(Q(0)\) (so both interest and inflation effects are
incorporated). Expressing differently, the real present value of a time-\(t\)
cash flow \(C\) is
\[
C\cdot\underbrace{\frac{Q(0)}{Q(t)}}_{\text{inflation}}\cdot\underbrace{v^t}_{\text{interest}}.
\]
\item Thus, given a price level index, we can simply modify present value to
\emph{real} present value to incorporate inflation. For example, basic formula
becomes
\[
P=Fr\qty(\frac{Q(0)}{Q(1)}v+\frac{Q(0)}{Q(2)}v^{2}+\dotsb+\frac{Q(0)}{Q(n)}v^{n})+C\cdot\frac{Q(0)}{Q(n)}v^n
\]
(assuming absence of taxes here; but income and capital gains taxes can also be
incorporated in a similar manner as before).
\end{enumerate}
\subsection{Bond Amortization}
\begin{enumerate}
\item Since bond \emph{seller} is effectively borrowing money from bond
\emph{buyer}, and the loan is repaid by installments (in the form of  coupon
payments and final payment of redemption value), we can also do \emph{loan
amortization} for a \emph{bond} (from the perspective of bond seller), like
\cref{subsect:amort-method}.

\item \label{it:book-value-prosp-fmla}
For bond amortization, an important quantity is \defn{book value} of a
bond at time \(k\) (\(k=0,\dotsc,n\)), which is the time-\(k\) price of the
bond (i.e., time-0 price of an otherwise identical bond but with \(n-k\)
periods term). Here, denote the time-\(k\) book value by \(B_k\). Then, by
basic formula, for any \(k=0,\dotsc,n\),
\begin{equation}
\label{eq:book-val-basic-fmla}
B_k=Fr\ax{\angl{n-k}}+Cv^{n-k}.
\end{equation}
\item To relate book value and loan amortization, the time-\(k\) book value can
actually be understood as the time-\(k\) \emph{outstanding balance} for the
loan (implied by the bond), by comparing \cref{eq:book-val-basic-fmla} with the
prospective method in loan amortization.

\begin{warning}
\Cref{eq:book-val-basic-fmla} is \underline{not} \emph{completely} identical to the formula
for prospective method. They are the same for any \(k=0,\dotsc,n-1\), but when
\(k=n\), the time-\(n\) book value here is \(C\) (while the time-\(n\)
outstanding balance is zero for loan amortization).
\end{warning}

\item \label{it:book-value-retro-recur-fmlas}
With this understanding, we can also develop similar formulas
corresponding to the recursive and retrospective methods in loan amortization:
\begin{note}
We need to have some special treatment for time-\(n\) book value, due to the
distinction mentioned above.
\end{note}
For any \(k=0,\dotsc,n\),
\begin{itemize}
\item (recursive) \(B_{k}=B_{k-1}(1+i)-Fr\)
\begin{warning}
We do \underline{not} have \(B_n=B_{n-1}(1+i)-Fr-C\) because of the special treatment!
\end{warning}
\item (retrospective) \(B_k=P(1+i)^k-Fr\sx{\angl{k}}\)
\begin{note}
\(P\) is the ``amount of loan'' in this context.
\end{note}
\begin{warning}
We do \underline{not} have \(B_n=P(1+i)^k-Fr\sx{\angl{n}}-C\) because of the special treatment!
\end{warning}
\end{itemize}

\item Now we introduce a notation to denote changes in book values: \(\Delta
B_k=B_{k+1}-B_k\), which is given by \(B_ki-Fr\) by recursive formula.

\item By premium/discount formula, for any \(k=0,\dotsc,n\), we have
\[
B_k=\underbrace{C}_{B_n}+\underbrace{C(g-i)\ax{\angl{n-k}}}_{\text{\ystar}}.
\]
From here we can observe that:
\begin{itemize}
\item When the bond is sold at premium (\(g>i\)), \ystar{} gradually
\faIcon{arrow-down} to zero (from a positive value) as \(k\) \faIcon{arrow-up}. Hence,
\(\Delta B_k<0\) for any \(k=1,\dotsc,n\).

\begin{remark}
\item \ystar{} is amount of premium (with respect to time-\(k\) price).
\item This process is known as \defn{amortization of premium} or \defn{writing down}.
\end{remark}

\item When the bond is sold at discount (\(g<i\)), \ystar{} gradually
\faIcon{arrow-up} to zero (from a negative value) as \(k\) \faIcon{arrow-up}. Hence,
\(\Delta B_k>0\) for any \(k=1,\dotsc,n\).

\begin{remark}
\item \ystar{} (in absolute value) is amount of discount (with respect to time-\(k\) price).
\item This process is known as \defn{accumulation of discount} or \defn{writing up}.
\end{remark}
\end{itemize}
In either case, the time-\(k\) book value gets closer and closer to the
time-\(n\) book value \(C\) as \(k\) \faIcon{arrow-up}. Intuitively, this is
because the premium is ``amortized'', or the discount is ``cancelled out'' by
``accumulation''.

\item \label{it:bond-amort-schedule}
We can also develop a \emph{bond amortization schedule}, like
\labelcref{it:amort-schedule}:
\begin{center}
\begin{tabular}{ccccc}
\toprule
Time \(k\)&
\makecell{``Installment''\\ (coupon) amount}&
\makecell{Interest repaid\\ \(I_k=B_{k-1}i\)}&
\makecell{Principal repaid \\ \(P_k=Cg-I_k\)}&Book value \(B_k\) \\
\midrule
0&\(0\)&\(0\)&\(0\)&\(C+C(g-i)\ax{\angl{n}}\) \\
1&\(Cg\)&\(B_0i\)&\(Cg-I_1\)&\(C+C(g-i)\ax{\angl{n-1}}\) \\
2&\(Cg\)&\(B_1i\)&\(Cg-I_2\)&\(C+C(g-i)\ax{\angl{n-2}}\) \\
\vdots&\vdots&\vdots&\vdots&\vdots\\
\(n\)&\(Cg\)&\(B_{n-1}i\)&\(Cg-I_{n}\)&\(C\)\\
\bottomrule
\end{tabular}
\end{center}

\item \label{it:bond-amort-pk-fmla}
From \labelcref{it:bond-amort-schedule}, we can derive a formula for
\(P_k\) for any \(k=1,\dotsc,n\):
\[
P_k=Cg-iB_{k-1}
=Cg-i\qty[C+C(g-i)\ax{n-k+1}]
=C(g-i)(1-i\ax{n-k+1})
=\boxed{C(g-i)v^{n-k+1}}.
\]
Since \(P_k={\color{violet}-}\Delta B_{k-1}\), this also suggests a formula for
\(\Delta B_k\) for any \(k=0,\dotsc,n-1\).
\end{enumerate}

\subsection{Serial Bonds}
\label{subsect:serial-bonds}
\begin{enumerate}
\item A \defn{serial bond} is a series of (possibly) coupon-paying
(``ordinary'') bonds with the same \(g\) and \(i\),  but possibly different
\(n\) (maturity dates differ) and \(C\) (redemption values differ).

\begin{note}
Thus, a serial bond is essentially a portfolio of (possibly) coupon-paying bonds.
\end{note}

\item \label{it:serial-bond-price-fmla}
To price a serial bond, we need to price the bonds in the series
individually and sum them up. To do this in a convenient way, we use
\emph{Makeham's formula}.

\item Suppose that there are \(m\) bonds in the series, with maturity dates
\(n_1,\dotsc,n_m\) and redemption values \(C_1,\dotsc,C_m\) respectively. We
also denote the present values of redemption values by \(K_1,\dotsc,K_m\), and
the (time-0) prices of \(m\) bonds by \(P_1,\dotsc,P_m\) respectively. Then,
the (time-0) price of serial bond is
\[
\sum_{i=1}^{m}P_i=\sum_{i=1}^{m}\qty(K_i+\frac{g}{i}(C_i-K_i))
=\sum_{i=1}^{m}K_i+\frac{g}{i}\qty(\sum_{i=1}^{m}C_i-\sum_{i=1}^{m}K_i)
=\boxed{K'+\frac{g}{i}(C'-K')}
\]
where \(\displaystyle K'=\sum_{i=1}^{m}K_i\) and \(\displaystyle
C'=\sum_{i=1}^{m}C_i\).

\begin{note}
The expression is still quite like the form in Makeham's formula: We just need
to use \(K'\) and \(C'\) instead of simply \(K\) and \(C\).
\end{note}
\end{enumerate}

\subsection{Callable Bonds}
\label{subsect:callable-bonds}
\begin{enumerate}
\item A \defn{callable bond} is a bond where the bond issuer (seller) has an
\emph{option} to redeem (i.e., pay the redemption value to buyer) early to
force the bond to mature at that time (``call'' the bond).

\begin{note}
Since there are different possible time for the issuer to call the bond, the
bond term (time length from time 0 to maturity) is \emph{not fixed} for a
callable bond.
\end{note}

\item To make the analysis of callable bond ``more tractable'', one way is to
assume that the option is utilized such that the result is \emph{optimal for
the issuer} (and hence worst for the buyer\footnote{This is because bond
transaction can be regarded as a \emph{zero-sum game} (ignoring transaction
costs etc.).}). So, from the perspective of \emph{bond buyer}, we carry out
calculations in ``worst-case scenario''.

\begin{note}
More precisely, ``worst result'' for bond buyer means the present value of cash
inflows (from the perspective of bond buyer) is the lowest.\footnote{Note that
the initial cash outflow (payment of bond price) can be ignored since it is the
same no matter when the issuer calls the bond: It has already been paid, and
changes in call timing would not affect that amount!}
\end{note}

\item \label{it:callable-bond-price}
A general first step for performing such calculations is to compare the
hypothetical purchasing prices that \emph{would} result under different
scenarios \begin{warning} Those hypothetical purchasing prices should not be
confused with the \emph{actual} purchasing price! \end{warning}
\faIcon{arrow-right} \emph{lowest one} is the worst for bond buyer since it
implies that under that scenario, the present value of all future cash inflows
is the lowest.

After that, the lowest value is the price of the callable bond since it
corresponds to the worst-case scenario for bond buyer.
\end{enumerate}
\subsection{Preferred Stocks}
\begin{enumerate}
\item A \defn{preferred stock} is a security which provides fixed dividend
payments at the end of each period forever.

\begin{note}
A preferred stock is like a bond, but it is an \emph{ownership security} rather
than a \emph{debt security}. Also, since the payment continues forever, it has
no maturity date.
\end{note}

\item \label{it:pref-stock-price}
Given a specific yield rate \(i\), the purchasing price \(P\) of a preferred stock
with dividend amount \(D\) can be found using a similar way as basic formula:
\[
P=D\ax{\angl{\infty}i}=\boxed{\frac{D}{i}}.
\]
\end{enumerate}
\subsection{Common Stocks}
\begin{enumerate}
\item A \defn{common stock} a security which provides dividend payments (which
are not fixed) at the end of each period forever.

\item The future dividends for a common stock are influenced by many factors
(e.g.\ profitability, economic environment etc.). Different people may have
different ``views'' on future dividends, even with the knowledge of the same
\emph{current information}. Still, an ``overall'' view from the market is
reflected by the current market price of the common stock. (See STAT3904 for
more discussions about this.)

\item \label{it:common-stock-price}
Suppose that a specific yield rate \(i\) is given. Then, consider a common
stock where the amounts of dividends at time 1, 2, 3, \(\dotsc\) are in a
geometric sequence: \(D\), \(D(1+k)\), \(D(1+k)^{2},\dotsc\) (resp.) with
\(-1<k<i\) (so that dividends are all positive and the geometric sum below is
finite).

The purchasing price \(P\) of the common stock is
\[
P=Dv\qty[1+\frac{1+k}{1+i}+\qty(\frac{1+k}{1+i})^{2}+\dotsb]
=\frac{Dv}{1-\frac{1+k}{1+i}}=\boxed{\frac{D}{i-k}}.
\]
\end{enumerate}
