\section{Evolution of Interest Rates}
\label{sect:int-evol}
\begin{enumerate}
\item In this section, we will focus on studying how interest rates/yield rates
\emph{evolve}, both over different \emph{bond terms}
(\cref{subsect:term-structure,subsect:spot-fwd-par-yld}) and over \emph{time}
(in a random manner) (\cref{subsect:stochastic-int}).
\end{enumerate}

\subsection{Term Structure of Interest Rates}
\label{subsect:term-structure}
\begin{enumerate}
\item In \cref{sect:bonds-other-securities}, we have assumed that yield rate
stays constant throughout, regardless of bond term.

However, this is clearly \emph{not} true in practice: We can observe that for
otherwise identical bonds with different terms, the (implied) yield rates
differ. This suggests that yield rate appears to have some relationship with
bond term \faIcon{arrow-right} a single value is \emph{not} sufficient to
describe yield rate!

\begin{note}
The phenomenon that yield rate varies according to bond term is called
\defn{term structure of interest rates}.
\end{note}

\item To represent the term structure of interest rates, we can use a
\defn{yield curve} (a function mapping from term to respective yield rate).
\begin{center}
\begin{tikzpicture}
\begin{axis}[domain=0.1:5, xmin=-0.2, ymin=-0.2, ymax=1.5,
axis lines=middle, ytick=\empty, xtick=\empty,
ylabel={(annual) yield rate},
xlabel={bond term}, title={Yield Curve}]
\addplot[blue]{1.2-1/(1+x)};
\end{axis}
\end{tikzpicture}
\end{center}
\item In attempt to \emph{explain} this phenomenon, there are three typical
theories: \begin{enumerate}
\item expectation theory
\item liquidity preference
\item market segmentation
\end{enumerate}
\item For \defn{expectation theory}:
\begin{itemize}
\item expected \faIcon{arrow-down} in future interest rate 

\faIcon{arrow-alt-circle-right} we want to ``capture'' high interest rate available now for \emph{longer time}

\faIcon{arrow-alt-circle-right} short-term (long-term) bond becomes {\color{red}\emph{less attractive}}
({\color{ForestGreen}\emph{more attractive}})

\faIcon{arrow-alt-circle-right} purchasing price \faIcon{arrow-down} (\faIcon{arrow-up})

\faIcon{arrow-alt-circle-right} implied yield rate \faIcon{arrow-up} (\faIcon{arrow-down})

\item expected \faIcon{arrow-up} in future interest rate 

\faIcon{arrow-alt-circle-right} we do not want to ``lock in'' a low interest rate available now for too long

\faIcon{arrow-alt-circle-right} short-term (long-term) bond becomes {\color{ForestGreen}\emph{more attractive}}
({\color{red}\emph{less attractive}})

\faIcon{arrow-alt-circle-right} purchasing price \faIcon{arrow-up} (\faIcon{arrow-down})

\faIcon{arrow-alt-circle-right} implied yield rate \faIcon{arrow-down} (\faIcon{arrow-up})
\end{itemize}

\item For \defn{liquidity preference}: Investors prefer to be ``liquid'' and
want to have free access to their funds

\faIcon{arrow-alt-circle-right} \emph{naturally} inclined towards short-term
rather than long-term bonds

\faIcon{arrow-alt-circle-right} \emph{higher} yield rate required for
\emph{long-term} bonds to compensate for this preference.

\item For \defn{market segmentation}: Bonds of \emph{different} terms are
attractive for \emph{different} investors (as they are used for different
purposes)

\faIcon{arrow-alt-circle-right} ``different markets'' for bonds of different
terms (market ``divided'' into multiple segments)

\faIcon{arrow-alt-circle-right} bonds of different terms are subject to
different forces of supply and demand

\faIcon{arrow-alt-circle-right} they have different prices and different yield
rates.
\end{enumerate}

\subsection{Spot Rates, Forward Rates, and Par Yields}
\label{subsect:spot-fwd-par-yld}
\begin{enumerate}
\item Apart from providing whole yield curve, there are other methods to describe
term structure of interest rates (partially), using the following terminologies:
\begin{enumerate}
\item spot rates
\item forward rates
\item par yields
\end{enumerate}
\begin{note}
The ``underlying'' measurement period for these rates is years by convention.
\end{note}

\item A \defn{spot rate} for a term of \(n\) periods (or \(n\)-period spot
rate) is the (constant) \emph{annual} interest rate equivalent to the
\(n\)-period interest rate (both rates are applicable from time 0 to time
\(n\)).  (Equivalently, it is the annual yield rate of an \(n\)-period
zero-coupon bond purchased at time 0\footnote{under the condition that the
interest rate earned in the zero-coupon bond is the same as the ``market''
interest rate (which should be assumed by default)}.) The notation for
\(n\)-period spot rate is \(s_n\).

\begin{note}
Typically \(n\) is an integer, but it does not have to be.
\end{note}

\item For example, if we are given \(n\)-period spot rate for every \(n\in\N\)
(for describing the term structure of interest rates), we can find the
purchasing price of a coupon-paying bond by basic formula as follows:
\[
P=Fr\qty[(1+s_1)^{-1}+(1+s_2)^{-2}+\dotsb+(1+s_n)^{-n}]+C(1+s_n)^{-n}.
\]
\item An \(m\)-period deferred \(n\)-period \defn{forward rate} is the
\emph{annual} interest rate equivalent to the \(n\)-period interest rate (both
rates are applicable from time \(m\) (starting time deferred for \(m\) periods)
to time \(m+n\)), denoted by \(\actsymb[n]{f}{m}\). \begin{warning}
This is \underline{not} to be confused with \(\actsymb[m]{f}{n}\)!
The right subscript denotes the \emph{starting time} and the left subscript
denotes the \emph{time length}.
\end{warning}

\begin{note}
Typically \(m\) and \(n\) are integers, but they do not have to be.
\end{note}

\item \label{it:fwd-rate-spot-rate-fmla}
Consider spot rates and forward rates for the \emph{same} term structure of
interest rates. Since the ``growth rate'' is supposed to be independent from
timing of initial investment (as in
\labelcref{it:growth-independent-amt-time}), we have the following
relationship between spot and forward rates:
\[
(1+\actsymb[n]{f}{m})^n=\frac{(1+s_{m+n})^{m+n}}{(1+s_{m})^{m}},
\]
for any \(m\) and \(n\).

\begin{center}
\begin{tikzpicture}
\draw[-Latex] (0,0) -- (10,0) node[right]{Time};
\fill[] (0,0) circle [radius=0.05]
node[below] {0}
node[above, brown]{1}
node[above=0.6cm, brown]{\faIcon{piggy-bank}};
\fill[] (4,0) circle [radius=0.05]
node[below=0.1cm] {\(m\)}
node[above, brown]{\((1+s_m)^{m}\)}
node[above=0.6cm, brown]{\faIcon{piggy-bank}}
node[below=0.6cm, violet]{\(1\)}
node[below=1.2cm, violet]{\faIcon{piggy-bank}};
\fill[] (8,0) circle [radius=0.05]
node[below=0.1cm] {\(m+n\)}
node[above, brown]{\((1+s_{m+n})^{m+n}\)}
node[above=0.6cm, brown]{\faIcon{piggy-bank}}
node[below=0.5cm, violet]{\((1+\actsymb[n]{f}{m})^{n}\)}
node[below=1.2cm, violet]{\faIcon{piggy-bank}};
\draw[-Latex, brown] (5,0.6) -- (7,0.6);
\draw[-Latex, violet] (5,-1.2) -- (7,-1.2);
\draw[<->, magenta] (6,-1.1) -- (6,0.5)
node[pos=0.4, right, text width=1cm]{same growth};

\node[brown] () at (10,1) {spot};
\node[violet] () at (10,-1.5) {forward};
\end{tikzpicture}
\end{center}

\item \label{it:spot-rate-in-fwd-rates-fmla}
As a corollary of \labelcref{it:fwd-rate-spot-rate-fmla}, we have
\[
(1+s_n)^n=(1+\actsymb[1]{f}{0})(1+\actsymb[1]{f}{1})\dotsb(1+\actsymb[1]{f}{n-1})
\]
for any \(n\in\N\).

\item Finally, for \emph{par yield}, it is related to a bond \emph{redeemable
at par} (i.e., \(C=F\)). To motivate the definition of par yield, we shall
consider the following result.
\begin{proposition}
\label{prp:price-par-coupon-yield}
For a bond redeemable at par, its coupon rate \(r\) equals its yield rate \(i\)
(assumed to be nonzero and exist uniquely) iff its purchasing price \(P\)
equals its par value \(F\).
\end{proposition}
\begin{pf}
``\(\Rightarrow\)'': Assume that \(r=i\). Then, by basic formula,
\[
P=Fi\ax{\angl{n}i}+Fv^n
=F(1-v^n)+Fv^n=F.
\]
``\(\Leftarrow\)'': Assume that \(P=F\). Then, by definition of IRR, we have
\[
-F+Fr\ax{\angl{n}i}+Fv^n=0\implies 
\frac{r}{i}(1-v^n)+v^n=1
\implies r(1-v^n)=i(1-v^n)
\implies r=i.
\]
\end{pf}
\item An \defn{\(n\)-period par yield} is a coupon rate for an \(n\)-period
bond that is redeemable at par such that its price equals its par value (``sold
at par''), denoted by \(i_{p_n}\) (\(n\in\N\)).

\begin{note}
By \cref{prp:price-par-coupon-yield}, equivalently, it is a coupon rate for a
bond redeemable at \underline{par} such that the coupon rate and
\underline{yield} rate coincide (when yield rate exists uniquely)
\faIcon{arrow-right} hence named ``par yield''.
\end{note}

\item Following the definition of par yield, given spot rates, we have by
basic formula
\[
F=Fi_{p_n}\qty[(1+s_1)^{-1}+\dotsb+(1+s_n)^{-n}]+F(1+s_n)^{-n},
\]
which implies
\[
i_{p_n}=\frac{1-(1+s_{n})^{-n}}{(1+s_1)^{-1}+\dotsb+(1+s_n)^{-n}}.
\]
From here we can see that par yield also suggests relationship of a kind of
``rate'' with bond term \(n\), ``through'' spot rates \faIcon{arrow-right} it
also describes term structure of interest rates.
\end{enumerate}
\subsection{Stochastic Approach to Interest}
\label{subsect:stochastic-int}
\begin{enumerate}
\item So far all interest-related quantities are deterministic (non-random)
\faIcon{arrow-right} \emph{deterministic} approach to interest.

\item Here we are interested in investigating the behaviour of various
quantities when we incorporate \emph{randomness} to interest
\faIcon{arrow-right} \emph{stochastic} approach to interest, which may be seen
as ``more realistic''.

\item We shall start by introducing randomness to \emph{effective} interest
rate for \(n\)th period: \(i_n\), for every \(n\in\N\). In stochastic approach
to interest, we shall suppose the effective interest rate for \(n\)th period is
a \emph{random variable} \(I_n\), for any \(n\in\N\).

\begin{note}
\(I_1,I_2,\dotsc\) may or may not be independent and they can have different
distributions.
\end{note}

\item Denote the mean and variance of \(I_n\) by \(j_n\) and \(s_n^2\)
respectively. Now, we are interested in what happens if we do the work in
\cref{sect:interest-measurement} under this stochastic setting.

\item \label{it:stoch-acc-value-prob-quantities-one-period}
For example, for the time-\(n\) accumulated value of 1 invested at time
\(n-1\):
\begin{itemize}
\item its mean is \(\expv{1+I_n}=\boxed{1+j_n}\);
\item its variance is \(\vari{1+I_n}=\vari{I_n}=\boxed{s_n^2}\);
\item its second moment is
\(\expv{(1+I_n)^{2}}=\vari{1+I_n}+(\expv{1+I_n})^{2}=\boxed{s_n^2+(1+j_n)^2}\).
\end{itemize}
\item \label{it:stoch-acc-value-prob-quantities-n-periods}
For calculations involving \emph{multiple} periods, they are affected by
the \emph{dependence structure} of \(I_1,I_2,\dotsc\). As a simple case, we
focus on the time interval \([0,n]\) and \(I_1,\dotsc,I_n\) are
\emph{independent}.

Denote the time-\(n\) accumulated value of principal of 1 by \(S_n\) (this is
\(a(n)\)). Then, by definition of effective interest rate, we have
\[
S_n=(1+I_1)\dotsb(1+I_n).
\]
Due to independence, we can compute the following probabilistic quantities
easily:
\begin{itemize}
\item mean: \(\expv{S_n}=\expv{1+I_1}\dotsb\expv{1+I_n}=\boxed{(1+j_1)\dotsb(1+j_n)}\)
\item second moment: \(\expv{S_n^2}=\expv{(1+I_1)^2}\dotsb\expv{(1+I_n)^2}
=\boxed{\qty[s_1^2+(1+j_1)^2]\dotsb\qty[s_n^2+(1+j_n)^2]}\)
\item variance: \(\vari{S_n}=\expv{S_n^2}-(\expv{S_n})^{2}
=\boxed{\qty[s_1^2+(1+j_1)^2]\dotsb\qty[s_n^2+(1+j_n)^2]-\qty[(1+j_1)\dotsb(1+j_n)]^{2}}\)
\end{itemize}

\item For further specification of the behaviour of stochastic interest, we may
impose a class of distribution on each of \(I_1,I_2,\dotsc\), and a common one
is \emph{lognormal distribution}.

\item A random variable \(Y\) follows a \defn{lognormal distribution} with
parameters \(\mu\) and \(\sigma^2\) (denoted by \(Y\sim
\text{\(LN(\mu,\sigma^2)\)}\)) if the ``log'' of \(Y\), \(\ln Y\),
follows a normal distribution with mean \(\mu\) and variance \(\sigma^2\),
i.e., \(\ln Y\sim N(\mu,\sigma^2)\).

\begin{warning}
The parameters \(\mu\) and \(\sigma^2\) are \emph{not} the mean and variance of
\(Y\)!
\end{warning}

\item \label{it:ln-dist-mean-var}
To compute moments of \(Y\), it is useful to recall that the moment
generating function of a normal r.v. \(X\sim N(\mu,\sigma^2)\):
\[
M_X(t)=\expv{e^{tX}}=\exp\qty(\mu t+\frac{1}{2}\sigma^2t^2).
\]
Since \(Y=e^{X}\), the first and second moments of \(Y\) are
\[
\expv{Y}=\expv{e^X}=M_X(1)=\boxed{e^{\mu+\sigma^2/2}},
\]
and
\[
\expv{Y^2}=\expv{e^{2X}}=M_X(2)=\boxed{e^{2(\mu+\sigma^2)}}.
\]
Hence, the variance of \(Y\) is
\[
\vari{Y}=\expv{Y^2}-\qty(\expv{Y})^{2}=\boxed{e^{2\mu+\sigma^2}\qty(e^{\sigma^2}-1)}.
\]
\item To impose lognormal distribution here, we shall assume \((1+I_n)\sim
LN(\mu_n,\sigma_n^2)\) (or equivalently, \(\ln(1+I_n)\sim
N(\mu_n,\sigma_n^2)\)) for any \(n\in\N\).

\begin{note}
Usage of lognormal distribution on \(1+I_n\) is ``reasonable'' here since
\(1+I_n\) should be positive (effective interest rate should be larger than
\(-100\%\), even if negative interest rate is possible!).
\end{note}
\item \label{it:lognormal-acc-value-sn-dist}
Again, we shall also assume that \(I_1,\dotsc,I_n\) are independent.
Then, as functions of \(I_1,\dotsc,I_n\),
\[\ln(1+I_1),\dotsc,\ln(1+I_n)\]
are also independent. Then, we have
\[
\ln S_n=\ln(\prod_{t=1}^{n}(1+I_t))
=\sum_{t=1}^{n}\ln(1+I_t)
\sim N\qty(\sum_{t=1}^{n}\mu_t,\sum_{t=1}^{n}\sigma_t^2),
\]
so in this case \(S_n\) is also lognormally distributed.

\begin{note}
After knowing the distribution of \(S_n\), we can use the formulas in
\labelcref{it:ln-dist-mean-var} to compute its mean and variance.
\end{note}
\end{enumerate}

