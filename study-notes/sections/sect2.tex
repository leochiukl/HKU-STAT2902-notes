\section{Interest Problems and Annuities}
\label{sect:int-prob-annuities}
\subsection{Interest Problems}
\begin{enumerate}
\item An interest problem (involving a single investment) generally involves
four basic quantities:
\begin{enumerate}
\item principal;
\item length of investment period;
\item interest rate (or force of interest);
\item accumulated value at the end of the investment period,
\end{enumerate}
where three of them are given, and we are asked to find the remaining one
(target quantity).

\item To find the target quantity, we need to:
\begin{enumerate}
\item set up an \emph{equation of
value}, i.e., equation in the form of
\[
\text{principal}\times\text{accumulation factor}=\text{accumulated value at the end}
\]
where ``accumulation factor'' incorporates the interest effect over the
investment period;
\item solve the equation for the target quantity, either analytically or
numerically.
\end{enumerate}
In simple cases, analytical method is possible for obtaining the solution. But
occasionally, numerical method is needed (e.g., when analytical method is too
``complex'' or does not exist).

\item Here we introduce a basic numerical method: \emph{linear interpolation}.
The process is as follows:
\begin{enumerate}
\item Set up the equation of value in the form of \(f(x)=0\) where \(x\) is the
unknown target quantity, for some function \(f\). (So, now we want to find
a \(x\) such that the equation holds, i.e., a \emph{root} of the equation.)
\item Find out two pairs of numbers \((x_1,f(x_1))\) and \((x_2,f(x_2))\) such
that \(f(x_1)<0<f(x_2)\). (They may be given in the question sometimes.)
\item Perform \emph{linear interpolation} between coordinates \((x_1,f(x_1))\)
and \((x_2,f(x_2))\): Joining them by a line segment and the intersection of
the segment and \(x\)-axis gives an \emph{approximated} root \(x^*\), which is
obtained by solving
\[
{\color{brown}\frac{f(x_2)-0}{x_2-x^*}}={\color{Salmon}\frac{f(x_2)-f(x_1)}{x_2-x_1}}.
\]
\begin{center}
\begin{tikzpicture}
\begin{axis}[domain=0:2.5, ymin=-0.9, ymax=0.9, axis x line=middle, axis y line=left,
extra x ticks={0.7}, extra x tick labels={\(x^*\)}, samples=100]
\addplot[blue, opacity=0.5]{-0.7*x^3+0.8*x^2+0.9*x-0.7};
\draw[violet, fill] (1,0.3) circle [radius=1.5pt]
node[above]{\((x_2,f(x_2))\)};
\draw[violet, fill] (0,-0.7) circle [radius=1.5pt]
node[right=0.3cm]{\((x_1,f(x_1))\)};
\draw[red, fill] (0.7,0) circle [radius=1.5pt];
\draw[violet] (0,-0.7) -- (1,0.3);
\draw[pen colour=brown, very thick, decorate,decoration={mirror, calligraphic brace, amplitude=5pt, raise=5pt}] (0.7,0) -- (1,0.3);
\draw[pen colour=Salmon, very thick, decorate,decoration={calligraphic brace, amplitude=5pt, raise=5pt}] (0.04,-0.66) -- (1,0.3);
\end{axis}
\end{tikzpicture}
\end{center}
\end{enumerate}
\end{enumerate}
\subsection{Method of Equated Time}
\begin{enumerate}
\item The \emph{method of equated time} is also an approach for
\emph{approximating} an unknown quantity, but in a setting where
\emph{multiple} investments are involved.
\item The setting is as follows. Suppose there are cash flows (e.g., amounts of
money withdrawn from several investments) of amounts \(C_1,\dotsc,C_n\) at time
\(t_1,\dotsc,t_n\) respectively.
\begin{center}
\begin{tikzpicture}
\draw[-Latex] (0,0) -- (10,0);
\draw[fill] (1,0) circle [radius=0.5mm];
\draw[fill] (3,0) circle [radius=0.5mm];
\draw[fill] (7,0) circle [radius=0.5mm];
\node[] () at (1,-0.5) {\(t_1\)};
\node[] () at (3,-0.5) {\(t_2\)};
\node[] () at (5,-0.5) {\(\cdots\)};
\node[] () at (7,-0.5) {\(t_n\)};

\node[] () at (1,0.5) {\(C_1\)};
\node[] () at (3,0.5) {\(C_2\)};
\node[] () at (5,0.5) {\(\cdots\)};
\node[] () at (7,0.5) {\(C_n\)};
\end{tikzpicture}
\end{center}
\begin{note}
Conventionally, ``cash flow'' of a nonnegative (negative) amount refers to cash
\underline{in}flow (\underline{out}flow) of that amount in absolute value.
\end{note}
\item Sometimes we want to have a ``sense'' of ``overall timing'' of these cash
flows (CFs). A metric for that is a time \(t\) such that
\[
\text{PV of a single CF \(C_1+\dotsb+C_n\) @ time \(t\)}
=
\text{PV of above CFs @ time \(t_1,\dotsc,t_n\)}\footnote{PV of multiple CFs is
obtained by summing the PV of each of them, since PV of multiple CFs means the
total amount needed now to accumulate (separately) to yield each of those CFs.},
\]
i.e.,
\[
(C_1+\dotsb+C_n)v^t=C_1v^{t_1}+\dotsb+C_nv^{t_n}.
\]
Here of course one can solve analytically for \(t\):
\[
t=\frac{1}{\ln v}\ln(\frac{C_1v^{t_1}+\dotsb+C_nv^{t_n}}{C_1+\dotsb+C_n}).
\]
However, the expression on RHS may appear to be ``too complex'' for some, and
the \emph{method of equated time} provides an approximation method that yields
a ``simpler'' expression.

\item \label{it:method-of-eq-time-fmla}
For \defn{method of equated time}, the solution \(t\) is approximated by
\[
\overline{t}=\frac{C_1t_1+\dotsb+C_nt_n}{C_1+\dotsb+C_n},
\]
which is a weighted average of time according to the amounts of cash flows.

\begin{note}
A ``better'' approximation that takes \emph{time value of money} into account
will be discussed in \cref{sect:duration}.
\end{note}
\end{enumerate}

\subsection{Annuity-Immediate and Annuity-Due}
\label{subsect:annuity-imm-due}
\begin{enumerate}
\item An \defn{annuity} is a series of cash inflows made at
\emph{equal} intervals of time (i.e., ``made regularly'').

\begin{remark}
\item An annuity whose payments are made \emph{certainly} is called \defn{annuity-certain}.
\item An annuity where \emph{uncertainties} are involved in the payments is
called \defn{contingent annuity}. (Special case: when payments are only made
while the recipient is \emph{alive}, the annuity is called \defn{life annuity}
(to be studied in STAT3901).)
\item We shall focus on annuity-certain in STAT2902: ``Annuity'' refers to
annuity-certain here.
\end{remark}
\item An \defn{\(n\)-period annuity-immediate} (\defn{\(n\)-period
annuity-due}) is an annuity whose cash inflows are made at the \emph{end}
(\emph{beginning}) of each period, for \(n\) periods.

\begin{note}
Here ``immediate'' is in the sense of ``immediately after current
\emph{period}'' (\underline{not} ``immediately after now''!): We need to wait
till the end of current period to receive the first cash inflow.

For ``due'', it is in the sense of ``due now'': The first cash flow is made now
(time 0).
\end{note}
\item \label{it:pv-annuity-imm}
The PV of an \(n\)-period annuity-immediate (amount of each CF = 1) is
\[
\ax{\angl{n}}=v+v^2+\dotsb+v^n=\frac{v}{1-v}(1-v^n)=\boxed{\frac{1-v^n}{i}}.
\]
\begin{center}
\begin{tikzpicture}
\draw[-Latex] (0,0) -- (12,0);
\draw[fill] (0,0) circle [radius=0.5mm];
\draw[fill] (2,0) circle [radius=0.5mm];
\draw[fill] (4,0) circle [radius=0.5mm];
\draw[fill] (8,0) circle [radius=0.5mm];
\draw[fill] (10,0) circle [radius=0.5mm];
\node[] () at (0,-0.5) {0};
\node[] () at (2,-0.5) {1};
\node[] () at (4,-0.5) {2};
\node[] () at (6,-0.5) {\(\cdots\)};
\node[] () at (8,-0.5) {\(n-1\)};
\node[] () at (10,-0.5) {\(n\)};
\node[] () at (2,0.5) {\(+1\)};
\node[] () at (4,0.5) {\(+1\)};
\node[] () at (8,0.5) {\(+1\)};
\node[] () at (10,0.5) {\(+1\)};
\end{tikzpicture}
\end{center}
\begin{note}
In general, for an \(n\)-period annuity-immediate (amount of each CF = \(k\)),
its PV is \(k\ax{\angl{n}}\). (Similar for other kinds of annuities.)
\end{note}

\item Before proceeding further, let us introduce a key ``trick'' that is
useful for deriving many upcoming formulas:
Accumulating/discounting (``shifting'') CFs at the given interest rate  \faIcon{arrow-right}
\emph{amount} of shifted CF becomes the \emph{value} of original CF at that time point.

The usefulness of this trick is that the value of the shifted CF turns out to
be identical to the one of the original CF (under \emph{compound} interest)!
This helps us ``reducing'' various situations to a familiar case.
\begin{center}
\begin{tikzpicture}
\draw[-Latex] (0,0) -- (12,0);
\draw[fill] (2,0) circle [radius=0.5mm];
\draw[fill] (6,0) circle [radius=0.5mm];
\node[] () at (2,0.5) {\(C\)};
\node[] () at (6,0.5) {\(C(1+i)^2\)};
\node[] () at (2,-0.5) {1};
\node[] () at (6,-0.5) {3};
\draw[-Latex, violet] (2.5,0.5) -- (5,0.5)
node[midway, above]{``forward shift''};
\end{tikzpicture}
\begin{tikzpicture}
\draw[-Latex] (0,0) -- (12,0);
\draw[fill] (2,0) circle [radius=0.5mm];
\draw[fill] (6,0) circle [radius=0.5mm];
\node[] () at (2,0.5) {\(Cv^2\)};
\node[] () at (6,0.5) {\(C\)};
\node[] () at (2,-0.5) {1};
\node[] () at (6,-0.5) {3};
\draw[-Latex, violet] (5.5,0.5) -- (2.5,0.5)
node[midway, above]{``backward shift''};
\end{tikzpicture}
\end{center}

\item \label{it:pv-annuity-due}
For the PV of an \(n\)-period annuity-due (amount of each CF = 1), we can
develop its formula based on \(\ax{\angl{n}}\) through ``shifting'':
\begin{center}
\begin{tikzpicture}
\draw[-Latex] (0,0) -- (12,0);
\draw[fill] (0,0) circle [radius=0.5mm];
\draw[fill] (2,0) circle [radius=0.5mm];
\draw[fill] (4,0) circle [radius=0.5mm];
\draw[fill] (8,0) circle [radius=0.5mm];
\draw[fill] (10,0) circle [radius=0.5mm];
\node[] () at (0,-0.5) {0};
\node[] () at (2,-0.5) {1};
\node[] () at (4,-0.5) {2};
\node[] () at (6,-0.5) {\(\cdots\)};
\node[] () at (8,-0.5) {\(n-1\)};
\node[] () at (10,-0.5) {\(n\)};
\node[] () at (0,0.5) {\(+1\)};
\node[] () at (2,0.5) {\(+1\)};
\node[] () at (4,0.5) {\(+1\)};
\node[] () at (8,0.5) {\(+1\)};
\node[] () at (6,1.5) {Original (annuity-due)};


\draw[-Latex] (0,-2) -- (12,-2);
\draw[fill] (0,-2) circle [radius=0.5mm];
\draw[fill] (2,-2) circle [radius=0.5mm];
\draw[fill] (4,-2) circle [radius=0.5mm];
\draw[fill] (8,-2) circle [radius=0.5mm];
\draw[fill] (10,-2) circle [radius=0.5mm];
\node[] () at (0,-2.5) {0};
\node[] () at (2,-2.5) {1};
\node[] () at (4,-2.5) {2};
\node[] () at (6,-2.5) {\(\cdots\)};
\node[] () at (8,-2.5) {\(n-1\)};
\node[] () at (10,-2.5) {\(n\)};
\node[] () at (2,-1.5) {\(+(1+i)\)};
\node[] () at (4,-1.5) {\(+(1+i)\)};
\node[] () at (8,-1.5) {\(+(1+i)\)};
\node[] () at (10,-1.5) {\(+(1+i)\)};
\node[] () at (6,-3.5) {Shifted (annuity-immediate)};

\draw[-Latex, violet] (0.5,0.2) -- (1.8,-1);
\draw[-Latex, violet] (2.5,0.2) -- (3.8,-1);
\draw[-Latex, violet] (4.5,0.2) -- (5.8,-1);
\draw[-Latex, violet] (6.5,0.2) -- (7.8,-1);
\draw[-Latex, violet] (8.5,0.2) -- (9.8,-1);
\end{tikzpicture}
\end{center}
Since value remains unchanged after ``shifting'', it follows that the desired
PV is
\[
\ax**{\angl{n}}=(1+i)\ax{\angl{n}}=(1+i)\frac{1-v^n}{i}=\boxed{\frac{1-v^n}{d}}.
\]
\begin{mnemonic}
The PV of an \(n\)-period annuity {\color{magenta}i}mmediate ({\color{magenta}d}ue) is
\(\displaystyle \frac{1-v^n}{{\color{magenta}i}}\) (\(\displaystyle \frac{1-v^n}{{\color{magenta}d}}\)).
\end{mnemonic}
\item After obtaining the formulas for the PV of \(n\)-period annuity-immediate
and annuity-due, getting the formulas for their \emph{accumulated value} at
time \(n\) becomes straightforward (through ``shifting'').
\item \label{it:av-annuity-imm}
For the accumulated value of annuity-immediate:
\begin{center}
\begin{tikzpicture}
\draw[-Latex] (0,0) -- (12,0);
\draw[fill] (0,0) circle [radius=0.5mm];
\draw[fill] (2,0) circle [radius=0.5mm];
\draw[fill] (4,0) circle [radius=0.5mm];
\draw[fill] (8,0) circle [radius=0.5mm];
\draw[fill] (10,0) circle [radius=0.5mm];
\node[] () at (0,-0.5) {0};
\node[] () at (2,-0.5) {1};
\node[] () at (4,-0.5) {2};
\node[] () at (6,-0.5) {\(\cdots\)};
\node[] () at (8,-0.5) {\(n-1\)};
\node[] () at (10,-0.5) {\(n\)};
\node[] () at (2,0.5) {\(+1\)};
\node[] () at (4,0.5) {\(+1\)};
\node[] () at (8,0.5) {\(+1\)};
\node[] () at (10,0.5) {\(+1\)};
\node[] () at (6,1.5) {Original (annuity-immediate)};


\draw[-Latex] (0,-2) -- (12,-2);
\draw[fill] (0,-2) circle [radius=0.5mm];
\draw[fill] (2,-2) circle [radius=0.5mm];
\draw[fill] (4,-2) circle [radius=0.5mm];
\draw[fill] (8,-2) circle [radius=0.5mm];
\draw[fill] (10,-2) circle [radius=0.5mm];
\node[] () at (0,-2.5) {0};
\node[] () at (2,-2.5) {1};
\node[] () at (4,-2.5) {2};
\node[] () at (6,-2.5) {\(\cdots\)};
\node[] () at (8,-2.5) {\(n-1\)};
\node[] () at (10,-2.5) {\(n\)};
\node[] () at (0,-1.5) {\(+\ax{\angl{n}}\)};
\node[] () at (6,-3.5) {Shifted (single CF)};

\draw[-Latex, violet] (1.8,0.2) -- (0.2,-1);
\draw[-Latex, violet] (3.8,0.2) -- (0.2,-1);
\draw[-Latex, violet] (5.8,0.2) -- (0.2,-1);
\draw[-Latex, violet] (7.8,0.2) -- (0.2,-1);
\draw[-Latex, violet] (9.8,0.2) -- (0.2,-1);
\end{tikzpicture}
\end{center}
From this we know that the accumulated value of an \(n\)-period annuity-immediate
at time \(n\) is the same as the one for the single CF:
\[
\sx{\angl{n}}=\ax{\angl{n}}(1+i)^n=\boxed{\frac{(1+i)^n-1}{i}}.
\]
\item \label{it:av-annuity-due}
For the accumulated value of annuity-due:
\begin{center}
\begin{tikzpicture}
\draw[-Latex] (0,0) -- (12,0);
\draw[fill] (0,0) circle [radius=0.5mm];
\draw[fill] (2,0) circle [radius=0.5mm];
\draw[fill] (4,0) circle [radius=0.5mm];
\draw[fill] (8,0) circle [radius=0.5mm];
\draw[fill] (10,0) circle [radius=0.5mm];
\node[] () at (0,-0.5) {0};
\node[] () at (2,-0.5) {1};
\node[] () at (4,-0.5) {2};
\node[] () at (6,-0.5) {\(\cdots\)};
\node[] () at (8,-0.5) {\(n-1\)};
\node[] () at (10,-0.5) {\(n\)};
\node[] () at (0,0.5) {\(+1\)};
\node[] () at (2,0.5) {\(+1\)};
\node[] () at (4,0.5) {\(+1\)};
\node[] () at (8,0.5) {\(+1\)};
\node[] () at (6,1.5) {Original (annuity-due)};


\draw[-Latex] (0,-2) -- (12,-2);
\draw[fill] (0,-2) circle [radius=0.5mm];
\draw[fill] (2,-2) circle [radius=0.5mm];
\draw[fill] (4,-2) circle [radius=0.5mm];
\draw[fill] (8,-2) circle [radius=0.5mm];
\draw[fill] (10,-2) circle [radius=0.5mm];
\node[] () at (0,-2.5) {0};
\node[] () at (2,-2.5) {1};
\node[] () at (4,-2.5) {2};
\node[] () at (6,-2.5) {\(\cdots\)};
\node[] () at (8,-2.5) {\(n-1\)};
\node[] () at (10,-2.5) {\(n\)};
\node[] () at (0,-1.5) {\(+\ax**{\angl{n}}\)};
\node[] () at (6,-3.5) {Shifted (single CF)};

\draw[-Latex, violet] (0,0.2) -- (0.2,-1);
\draw[-Latex, violet] (1.8,0.2) -- (0.2,-1);
\draw[-Latex, violet] (3.8,0.2) -- (0.2,-1);
\draw[-Latex, violet] (5.8,0.2) -- (0.2,-1);
\draw[-Latex, violet] (7.8,0.2) -- (0.2,-1);
\end{tikzpicture}
\end{center}
Likewise, the accumulated value of an \(n\)-period annuity-due at time \(n\) is
\[
\sx**{\angl{n}}=\ax**{\angl{n}}(1+i)^n=\boxed{\frac{(1+i)^n-1}{d}}.
\]
\end{enumerate}
\subsection{Perpetuities}
\label{subsect:perpetuity}
\begin{enumerate}
\item Simply speaking, a \emph{perpetuity} is an ``\(\infty\)-period'' annuity.
\item A \defn{perpetuity-immediate} (\defn{perpetuity-due}) is an
annuity-immediate (annuity-due) whose cash inflows last forever.
\item \label{it:pv-perpetuity-imm}
The PV of a perpetuity-immediate is then the limit
\[
\ax{\angl{\infty}}=\sum_{n=1}^{\infty}v^n=\lim_{n\to \infty}\ax{\angl{n}}
=\lim_{n\to \infty}\frac{1-v^n}{i}=\boxed{\frac{1}{i}}.
\]
\item \label{it:pv-perpetuity-due}
Likewise, the PV of a perpetuity-due is
\[
\ax**{\angl{\infty}}=\sum_{n=0}^{\infty}v^n=\lim_{n\to \infty}\ax**{\angl{n}}
=\lim_{n\to \infty}\frac{1-v^n}{d}=\boxed{\frac{1}{d}}.
\]
\end{enumerate}
\subsection{Annuities With CFs Less Frequent Than ``Each Period''}
\label{subsect:annuities-less-freq}
\begin{enumerate}
\item Consider here annuities with CFs made \emph{each \(k\) periods}, where
\(k>1\) is an integer. In this case, we can also develop formulas for
annuity/perpetuity-immediate and annuity/perpetuity-due, by ``adjusting'' the
formulas mentioned in \cref{subsect:annuity-imm-due,subsect:perpetuity}.
\item First, we discuss \(n\)-period annuities with such CF frequency. Here we
shall assume that \(n\) is divisible by \(k\) (so that \(n/k\in\N\)).
\item For an \(n\)-period annuity-immediate (due) with such CF frequency, the cash
inflows are made at the end (beginning) of \emph{each \(k\) periods}, for \(n\)
periods. (So there are \(n/k\in\N\) cash inflows in total.)

\begin{note}
For a perpetuity-immediate (due) with such CF frequency, it is defined in a
similar manner as before: the respective annuity-immediate (due) with cash
inflows lasting forever.
\end{note}
\item An important trick to derive formulas in this kind of case (different CF
frequencies) is called \defn{\(j\)-method}. The procedure is as follows:
\begin{enumerate}
\item Find the \emph{\(k\)-period} interest rate \(j\) that is equivalent to the
(1-period) interest rate \(i\) (by solving \(1+j=(1+i)^k\) for \(j\)).
\item Change the measurement period to \(k\) periods (1 unit of time = \(k\)
periods now).
\item Apply the formulas derived in
\cref{subsect:annuity-imm-due,subsect:perpetuity} with interest rate \(j\).
\begin{warning}
Check the amount of each CF to see if multiplication of a constant is needed!
\end{warning}
\end{enumerate}
A key benefit of this method is that it does \emph{not} require us to memorize
another set of formulas to handle calculations in this kind of situation.
Hence, this should be the main approach to be used in solving problems of this
kind.
\item Occasionally, specific ``formulas'' for this kind of situation are useful
(and provide some ``shortcuts''). So, henceforth we will derive some formulas
(still based on \(j\)-method mainly) for the sake of completeness.
\item \label{it:pv-annuity-imm-less-freq}
For the PV of an \(n\)-period annuity-immediate with such CF frequency (amount of each CF = 1):
\begin{center}
\begin{tikzpicture}
\draw[-Latex] (0,0) -- (12,0);
\draw[fill] (0,0) circle [radius=0.5mm];
\draw[fill] (2,0) circle [radius=0.5mm];
\draw[fill] (4,0) circle [radius=0.5mm];
\draw[fill] (8,0) circle [radius=0.5mm];
\draw[fill] (10,0) circle [radius=0.5mm];
\node[] () at (0,-0.5) {0};
\node[] () at (2,-0.5) {\(k\)};
\node[] () at (4,-0.5) {\(2k\)};
\node[] () at (6,-0.5) {\(\cdots\)};
\node[] () at (8,-0.5) {\(n-k\)};
\node[] () at (10,-0.5) {\(n\)};
\node[] () at (2,0.5) {\(+1\)};
\node[] () at (4,0.5) {\(+1\)};
\node[] () at (8,0.5) {\(+1\)};
\node[] () at (10,0.5) {\(+1\)};
\node[draw] () at (6,1.5) {interest rate per unit of time: \(i\)};
\end{tikzpicture}
\end{center}
Now, using \(j\)-method, we change the measurement period (divide each of the
time above by \(k\)) and get:
\begin{center}
\begin{tikzpicture}
\draw[-Latex] (0,0) -- (12,0);
\draw[fill] (0,0) circle [radius=0.5mm];
\draw[fill] (2,0) circle [radius=0.5mm];
\draw[fill] (4,0) circle [radius=0.5mm];
\draw[fill] (8,0) circle [radius=0.5mm];
\draw[fill] (10,0) circle [radius=0.5mm];
\node[] () at (0,-0.5) {0};
\node[] () at (2,-0.5) {1};
\node[] () at (4,-0.5) {2};
\node[] () at (6,-0.5) {\(\cdots\)};
\node[] () at (8,-0.5) {\(\displaystyle \frac{n}{k}-1\)};
\node[] () at (10,-0.5) {\(\displaystyle \frac{n}{k}\)};
\node[] () at (2,0.5) {\(+1\)};
\node[] () at (4,0.5) {\(+1\)};
\node[] () at (8,0.5) {\(+1\)};
\node[] () at (10,0.5) {\(+1\)};
\node[draw] () at (6,1.5) {interest rate per unit of time: \(j\)};
\end{tikzpicture}
\end{center}
Applying the equivalency relation between \(i\) and \(j\), we can get its PV:
\[
\ax{\angl{n/k}j}=\frac{1-(1+j)^{-n/k}}{j}
=\frac{1-(1+i)^{-n}}{(1+i)^k-1}=\frac{1-v^n}{i}\cdot\frac{i}{(1+i)^k-1}
=\boxed{\frac{\ax{\angl{n}i}}{\sx{\angl{k}i}}}.
\]
\begin{note}
The ``\(j\)'' in the notation is to signify the interest rate (per unit of
time) used is \(j\).
\end{note}

\item \label{it:pv-annuity-due-less-freq}
For the PV of an \(n\)-period annuity-due with such CF frequency (amount of
each CF = 1), applying the \(j\)-method and then using the previous relation
with annuity-immediate (``shifting'') gives its PV:
\[
\ax**{\angl{n/k}j}=(1+j)\ax{\angl{n/k}j}=(1+i)^{k}\frac{\ax{\angl{n}i}}{\sx{\angl{k}i}}
=\boxed{\frac{\ax{\angl{n}i}}{\ax{\angl{k}i}}}.
\]
\item\label{it:pv-perpetuities-less-freq}
Next, we can easily obtain formulas for PVs of perpetuity-immediate and
perpetuity-due with such CF frequency (amount of each CF = 1):
\begin{itemize}
\item perpetuity-immediate: \(\displaystyle \text{PV}=\lim_{n\to \infty}\frac{\ax{\angl{n}i}}{\sx{\angl{k}i}}
=\frac{\ax{\angl{\infty}i}}{\sx{\angl{k}i}}=\boxed{\frac{1}{i\sx{\angl{k}i}}}\);
\item perpetuity-due: \(\displaystyle \text{PV}=\lim_{n\to \infty}\frac{\ax{\angl{n}i}}{\ax{\angl{k}i}}
=\frac{\ax{\angl{\infty}i}}{\ax{\angl{k}i}}=\boxed{\frac{1}{i\ax{\angl{k}i}}}\).
\end{itemize}
\end{enumerate}
\subsection{Annuities With CFs More Frequent Than ``Each Period''}
\begin{enumerate}
\item Consider here annuities with CFs made \emph{each \(1/m\) of a period}, where
\(m>1\) is an integer. Likewise we can use \(j\)-method in this case \begin{warning}
Be careful about the amount of each CF!
\end{warning}, and we shall derive some formulas based on it (which are more
commonly used than the ones in \cref{subsect:annuities-less-freq}).

\item For an \(n\)-period annuity-immediate (due) with such CF frequency, the
cash inflows are made at the end (beginning) of \emph{each \(1/m\) of a
period}, for \(n\) periods. (So there are \(mn\in\N\) cash inflows in total.)

\item \label{it:pv-av-annuity-imm-more-freq}
For an \(n\)-period annuity-immediate with such CF frequency (amount of each CF
= \(1/m\) \warn):
\begin{center}
\begin{tikzpicture}
\draw[-Latex] (0,0) -- (12,0);
\draw[fill] (0,0) circle [radius=0.5mm];
\draw[fill] (2,0) circle [radius=0.5mm];
\draw[fill] (4,0) circle [radius=0.5mm];
\draw[fill] (8,0) circle [radius=0.5mm];
\draw[fill] (10,0) circle [radius=0.5mm];
\node[] () at (0,-0.5) {0};
\node[] () at (2,-0.5) {\(1/m\)};
\node[] () at (4,-0.5) {\(2/m\)};
\node[] () at (6,-0.5) {\(\cdots\)};
\node[] () at (8,-0.5) {\(n-1/m\)};
\node[] () at (10,-0.5) {\(n\)};
\node[] () at (2,0.5) {\(+1/m\)};
\node[] () at (4,0.5) {\(+1/m\)};
\node[] () at (8,0.5) {\(+1/m\)};
\node[] () at (10,0.5) {\(+1/m\)};
\node[draw] () at (6,1.5) {interest rate per unit of time: \(i\)};
\end{tikzpicture}
\end{center}
Now again we change the measurement period:
\begin{center}
\begin{tikzpicture}
\draw[-Latex] (0,0) -- (12,0);
\draw[fill] (0,0) circle [radius=0.5mm];
\draw[fill] (2,0) circle [radius=0.5mm];
\draw[fill] (4,0) circle [radius=0.5mm];
\draw[fill] (8,0) circle [radius=0.5mm];
\draw[fill] (10,0) circle [radius=0.5mm];
\node[] () at (0,-0.5) {0};
\node[] () at (2,-0.5) {1};
\node[] () at (4,-0.5) {2};
\node[] () at (6,-0.5) {\(\cdots\)};
\node[] () at (8,-0.5) {\(mn-1\)};
\node[] () at (10,-0.5) {\(mn\)};
\node[] () at (2,0.5) {\(+1/m\)};
\node[] () at (4,0.5) {\(+1/m\)};
\node[] () at (8,0.5) {\(+1/m\)};
\node[] () at (10,0.5) {\(+1/m\)};
\node[draw] () at (6,1.5) {interest rate per unit of time: \(j\)};
\end{tikzpicture}
\end{center}
\begin{note}
To get the interest rate \(j\), we solve \(1+j=(1+i)^{1/m}\) for \(j\).
\end{note}

Hence, the PV is
\[
\ax{\angl{n}}[(m)]
=\frac{1}{m}\ax{\angl{mn}j}
=\frac{1}{m}\cdot\frac{1-(1+j)^{-mn}}{j}
=\frac{1-(1+i)^{-n}}{m\qty[(1+i)^{1/m}-1]}
=\boxed{\frac{1-v^n}{i^{(m)}}}.
\]
Thus, by ``shifting'', the time-\(n\) accumulated value is
\[
\sx{\angl{n}}[(m)]
=\ax{\angl{n}}[(m)]\underbrace{(1+i)^n}_{\mathclap{\text{or }(1+j)^{mn}}}
=\boxed{\frac{(1+i)^n-1}{i^{(m)}}}.
\]
\item \label{it:pv-av-annuity-due-more-freq}
For an \(n\)-period annuity-due with such CF frequency (amount of each CF =
\(1/m\)), we can likewise apply the \(j\)-method and use the previous relation
with annuity-immediate to get its PV:
\[
\ax**{\angl{n}}[(m)]=\frac{1}{m}\ax**{\angl{mn}j}
=\frac{1}{m}(1+j)\ax{\angl{mn}j}
=\underbrace{(1+j)}_{(1+i)^{1/m}}\cdot\frac{1-v^n}{i^{(m)}}
=\frac{1-v^n}{(1+i)^{-1/m}\cdot \underbrace{m[(1+i)^{1/m}-1]}_{i^{(m)}}}
=\boxed{\frac{1-v^n}{d^{(m)}}}
\]
since \(\displaystyle \qty(1-\frac{d^{(m)}}{m})^{-m}=1+i\implies d^{(m)}=m\qty[1-(1+i)^{-1/m}]\).

Similarly, its time-\(n\) accumulated value is
\[
\sx**{\angl{n}}[(m)]=\ax**{\angl{n}}[(m)](1+i)^{n}=\boxed{\frac{(1+i)^n-1}{d^{(m)}}}.
\]
\begin{mnemonic}
The PV of an \(n\)-period annuity {\color{magenta}i}mmediate
({\color{magenta}d}ue) (amount of each CF = \(1/m\)) with such CF frequency is
\(\displaystyle \frac{1-v^n}{{\color{magenta}i^{(m)}}}\) (\(\displaystyle
\frac{1-v^n}{{\color{magenta}d^{(m)}}}\)).
\end{mnemonic}

\item \label{it:pv-perpetuities-more-freq}
Next, we can similarly obtain formulas for PVs of perpetuity-immediate and
perpetuity-due with such CF frequency (amount of each CF = \(1/m\)):
\begin{itemize}
\item perpetuity-immediate: \(\displaystyle \text{PV}=\lim_{n\to \infty}\ax{\angl{n}}[(m)]
=\boxed{\frac{1}{i^{(m)}}}\);
\item perpetuity-due: \(\displaystyle \text{PV}=\lim_{n\to \infty}\ax**{\angl{n}}[(m)]
=\boxed{\frac{1}{d^{(m)}}}\).
\end{itemize}
\end{enumerate}
\subsection{Continuous Annuities}
\label{subsect:cts-annuity}
\begin{enumerate}
\item Loosely speaking, a \emph{continuous annuity} is an annuity with
``infinitesimal'' CFs made ``\emph{each \(1/\infty\) of a period}''
(\emph{continuously}). The ``total amount'' of those ``infinitesimal'' CFs for
one period is the \defn{rate of CFs}.
\item \label{it:pv-av-cts-annuity}
Under \emph{constant} force of interest \(\delta\) (equivalent to the rate
\(i\)), for an \(n\)-period continuous annuity (rate of cash inflows = 1), the
PV can be obtained as the limit
\[
\ax*{\angl{n}}=\lim_{m\to \infty}\ax{\angl{n}}[(m)]=\lim_{m\to \infty}\frac{1-v^n}{i^{(m)}}
=\boxed{\frac{1-v^n}{\delta}}.
\]
\begin{remark}
\item The nominal rate \(i^{(m)}\) is equivalent to the rate \(i\) for every \(m\),
so it approaches \(\delta\) as \(m\to\infty\).
\item This is also equal to the limit \(\lim_{m\to \infty}\ax**{\angl{n}}[(m)]\) 
as we also have \(\lim_{m\to \infty}d^{(m)}=\delta\) (where the nominal rate
\(d^{(m)}\) is equivalent to the rate \(i\) for every \(m\)).
\end{remark}

For the time-\(n\) accumulated value, applying ``shifting'' argument to the
expression in the limit gives
\[
\sx*{\angl{n}}=\lim_{m\to \infty}\ax{\angl{n}}[(m)](1+i)^n
=\ax*{\angl{n}}(1+i)^n
=\boxed{\frac{(1+i)^n-1}{\delta}}.
\]

\item \label{it:pv-varying-foi-cts-annuity}
In general, under \emph{varying} force of interest \(\delta_t\) (as
a function of time \(t\)), the formula for PV of an \(n\)-period continuous
annuity (rate of cash inflows = 1) is given by
\[
\int_{0}^{n}a^{-1}(t)\dd{t}
=\int_{0}^{n}\exp(-\int_{0}^{t}\delta_s\dd{s})\dd{t}
\]
(which may be seen as following from the definition).

\begin{intuition}
In each infinitesimal time interval \([t,t+\dd{t}]\), the PV of the ``stream''
of cash inflows during the interval (total amount: \(\text{rate}\times\text{time length}=1\times \dd{t}=\dd{t}\)) is
\[{\color{brown}\text{total amount}}\times{\color{violet}\text{discount factor}}
={\color{violet}a^{-1}(t)}{\color{brown}\dd{t}}.\]
``Summing'' all the PVs up then gives the total PV:
\[
\int_{0}^{n}a^{-1}(t)\dd{t}.
\]
\end{intuition}
\end{enumerate}
\subsection{Annuities With CFs Varying in Arithmetic Sequence}
\begin{enumerate}
\item Even among annuities with CFs varying in arithmetic sequence, we can
divide them into different groups (by the frequency of CFs). In this section we
will discuss the following groups:
\begin{itemize}
\item ``each period'' (one CF per period)
\item ``each \(k\) periods'' (``less frequent'')
\item ``each \(1/m\) of a period'' (``more frequent'')
\end{itemize}
We shall focus on \emph{annuity-immediate} here.
\begin{note}
The formulas (notations) can be easily translated for the annuity-due case, by
``shifting'' (adding two dots on top of \(a\)).
\end{note}
\subsubsection*{``Each Period''}
\item \label{it:annuity-as-gen-form}
In this case, for an \(n\)-period annuity-immediate with such CF pattern,
the cash inflows at time \(1,\dotsc,n\) are \(P, P+Q, P+2Q,\dotsc,P+(n-1)Q\)
respectively (where the constants \(P\) and \(Q\) are such that all quantities
involved are nonnegative).
\begin{center}
\begin{tikzpicture}
\draw[-Latex] (0,0) -- (12,0);
\draw[fill] (0,0) circle [radius=0.5mm];
\draw[fill] (2,0) circle [radius=0.5mm];
\draw[fill] (4,0) circle [radius=0.5mm];
\draw[fill] (8,0) circle [radius=0.5mm];
\draw[fill] (10,0) circle [radius=0.5mm];
\node[] () at (0,-0.5) {0};
\node[] () at (2,-0.5) {1};
\node[] () at (4,-0.5) {2};
\node[] () at (6,-0.5) {\(\cdots\)};
\node[] () at (8,-0.5) {\(n-1\)};
\node[] () at (10,-0.5) {\(n\)};
\node[] () at (2,0.5) {\(+P\)};
\node[] () at (4,0.5) {\(+P+Q\)};
\node[font=\small] () at (7.5,0.5) {\(+P+(n-2)Q\)};
\node[font=\small] () at (10.5,0.5) {\(+P+(n-1)Q\)};
\end{tikzpicture}
\end{center}
\item To find its PV, we use the following result
\begin{proposition}
\label{prp:pv-as-annuity-imm}
The PV of this kind of \(n\)-period annuity-immediate is
\[
P\ax{\angl{n}}+Q\frac{\ax{\angl{n}}-nv^n}{i}.
\]
\end{proposition}
\begin{pf}
Firstly, the PV is
\begin{align*}
&Pv+(P+Q)v^{2}+(P+2Q)v^{3}+\dotsb+(P+(n-1)Q)v^{n} \\
&\quad=P(v+v^2+v^3+\dotsb+v^n)+Q(\underbrace{v^2+2v^3+\dotsb+(n-1)v^n}_{u}) \\
&\quad=P\ax{\angl{n}}+Qu.
\end{align*}
It then suffices to show that \(\displaystyle u=\frac{\ax{\angl{n}}-nv^n}{i}\).
Now, we use the following trick:
\begin{align*}
(1+i)u-u = v&+2v^2+3v^3+\dotsb+(n-1)v^{n-1}\\
& -v^2-2v^3-\dotsb-(n-2)v^{n-1}{\color{violet}-(n-1)v^n} \\
=v&+v^2+v^3+\dotsb+v^{n-1}+{\color{violet}v^n-nv^n}
=\ax{\angl{n}}-nv^n,
\end{align*}
as desired.
\end{pf}

\item \label{it:pv-av-incr-decr-annuities}
Now we introduce some special cases (with names and special notations):
\begin{itemize}
\item \defn{\(n\)-period increasing annuity} (\(P=1, Q=1\)):
\[
\text{PV}=\Ia_{\angl{n}}=
\ax{\angl{n}}+\frac{\ax{\angl{n}}-nv^n}{i}
=\frac{(1+i)\ax{\angl{n}}-nv^n}{i}
=\boxed{\frac{\ax**{\angl{n}}-nv^n}{i}},
\]
and (by ``shifting'')
\[
\text{time-\(n\) AV}=\Is_{\angl{n}}=\Ia_{\angl{n}}(1+i)^n=\boxed{\frac{\sx**{\angl{n}}-n}{i}}.
\]
\item \defn{\(n\)-period decreasing annuity} (\(P=n, Q=-1\)):
\[
\text{PV}=\Da_{\angl{n}}=
n\underbrace{\ax{\angl{n}}}_{\mathclap{(1-v^n)/i}}-\frac{\ax{\angl{n}}-nv^n}{i}
=\frac{n-nv^n-\ax{\angl{n}}+nv^n}{i}
=\boxed{\frac{n-\ax{\angl{n}}}{i}},
\]
and (by ``shifting'')
\[
\text{time-\(n\) AV}=\Ds_{\angl{n}}=\Da_{\angl{n}}(1+i)^n=\boxed{\frac{n(1+i)^n-\sx{\angl{n}}}{i}}.
\]
\end{itemize}

\item We can obtain another set of PV formulas for the increasing and
decreasing annuities using ``horizontal view'' and ``splitting''.

\item \label{it:pv-incr-annuity-alt}
For increasing annuity:
\begin{center}
\begin{tikzpicture}
\draw[-Latex] (0,0) -- (12,0);
\draw[fill] (0,0) circle [radius=0.5mm];
\draw[fill] (2,0) circle [radius=0.5mm];
\draw[fill] (4,0) circle [radius=0.5mm];
\draw[fill] (8,0) circle [radius=0.5mm];
\draw[fill] (10,0) circle [radius=0.5mm];
\node[] () at (0,-0.5) {0};
\node[] () at (2,-0.5) {1};
\node[] () at (4,-0.5) {2};
\node[] () at (6,-0.5) {\(\cdots\)};
\node[] () at (8,-0.5) {\(n-1\)};
\node[] () at (10,-0.5) {\(n\)};
\node[brown] () at (2,0.5) {\(+1\)};
\node[brown] () at (4,0.5) {\(+1\)};
\node[brown] () at (8,0.5) {\(+1\)};
\node[brown] () at (10,0.5) {\(+1\)};
\node[violet] () at (4,1) {\(+1\)};
\node[violet] () at (8,1) {\(+1\)};
\node[violet] () at (10,1) {\(+1\)};
\node[] () at (8,1.5) {\(\vdots\)};
\node[] () at (10,1.5) {\(\vdots\)};
\node[blue] () at (8,2) {\(+1\)};
\node[blue] () at (10,2) {\(+1\)};
\node[magenta] () at (10,2.5) {\(+1\)};
\end{tikzpicture}
\end{center}
From this we can express the PV of an \(n\)-period increasing annuity as
\[
{\color{brown}\ax{\angl{n}}}
+{\color{violet}v\ax{\angl{n-1}}}
+\dotsb+
+{\color{blue}v^{n-1}\ax{\angl{2}}}
+{\color{magenta}v^{n}\ax{\angl{1}}}
=\sum_{t=0}^{{\color{Maroon}n-1}}v^t\ax{\angl{n-t}}
\]
(by ``shifting'' cash inflows in each row such that the first one is at time 1).

\item \label{it:pv-decr-annuity-alt}
For decreasing annuity:
\begin{center}
\begin{tikzpicture}
\draw[-Latex] (0,0) -- (12,0);
\draw[fill] (0,0) circle [radius=0.5mm];
\draw[fill] (2,0) circle [radius=0.5mm];
\draw[fill] (4,0) circle [radius=0.5mm];
\draw[fill] (8,0) circle [radius=0.5mm];
\draw[fill] (10,0) circle [radius=0.5mm];
\node[] () at (0,-0.5) {0};
\node[] () at (2,-0.5) {1};
\node[] () at (4,-0.5) {2};
\node[] () at (6,-0.5) {\(\cdots\)};
\node[] () at (8,-0.5) {\(n-1\)};
\node[] () at (10,-0.5) {\(n\)};
\node[brown] () at (2,0.5) {\(+1\)};
\node[brown] () at (4,0.5) {\(+1\)};
\node[brown] () at (8,0.5) {\(+1\)};
\node[brown] () at (10,0.5) {\(+1\)};
\node[violet] () at (2,1) {\(+1\)};
\node[violet] () at (4,1) {\(+1\)};
\node[violet] () at (8,1) {\(+1\)};
\node[] () at (2,1.5) {\(\vdots\)};
\node[] () at (4,1.5) {\(\vdots\)};
\node[blue] () at (2,2) {\(+1\)};
\node[blue] () at (4,2) {\(+1\)};
\node[magenta] () at (2,2.5) {\(+1\)};
\end{tikzpicture}
\end{center}
From this we can likewise express the PV of an \(n\)-period decreasing annuity as
\[
{\color{brown}\ax{\angl{n}}}
+{\color{violet}\ax{\angl{n-1}}}
+\dotsb+
+{\color{blue}\ax{\angl{2}}}
+{\color{magenta}\ax{\angl{1}}}
=\sum_{t=0}^{{\color{Maroon}n-1}}\ax{\angl{n-t}}.
\]
\begin{note}
Here we do not even need ``shifting''!
\end{note}

\item \label{it:pv-as-perpetuity-imm}
We can also develop a formula for \emph{perpetuity-immediate} with the
CF pattern in \labelcref{it:annuity-as-gen-form}, i.e., cash inflows at time
\(1,2,\dotsc,\) are \(P,P+Q,\dotsc\) respectively:
\[
\text{PV}=\lim_{n\to \infty}\qty(P\ax{\angl{n}}+Q\cdot\frac{\ax{\angl{n}}-nv^n}{i})
=\frac{P}{i}+\frac{Q\ax{\angl{\infty}}}{i}
=\boxed{\frac{P}{i}+\frac{Q}{i^2}}.
\]
\item As a special case, the PV of an \defn{increasing perpetuity} (increasing
annuity whose cash inflows last forever) is
\[
\Ia_{\angl{\infty}}=\lim_{n\to \infty}\frac{\ax**{\angl{n}}-nv^n}{i}
=\frac{\ax**{\angl{\infty}}}{i}
=\frac{1}{id}.
\]
\begin{remark}
\item An intuitive reason for \(\lim_{n\to \infty}nv^n=0\) is that as \(n\to\infty\),
the ``speed'' at which \(v^n\) drops to zero (exponential) is higher than the
``speed'' at which \(n\) approaches infinity (linear). Formally, one can use
L'H\^{o}pital's rule to show this.
\item There is not ``decreasing perpetuity'' since negative cash
inflow is not allowed.
\end{remark}

\subsubsection*{``Each \(k\) Periods''}
\item Here consider an \(n\)-period annuity-immediate where the cash inflows at
time \(k,2k,\dotsc,n\) are \(1, 2,\dotsc,n/k\) respectively (assuming \(n\) is
divisible by \(k\)).
\item \label{it:pv-annuity-incr-each-k}
To determine its PV, we can use \(j\)-method (the \(k\)-period rate \(j\) is
found by solving \(1+j=(1+i)^k\)) and apply the formula for PV of \(n\)-period
increasing annuity in \labelcref{it:pv-av-incr-decr-annuities}:
\[
\text{PV}=(Ia)_{\angl{n/k}j}=\frac{\ax**{\angl{n/k}j}-(n/k)(1+j)^{-n/k}}{j}
=\boxed{\frac{\frac{\ax{\angl{n}i}}{\ax{\angl{k}i}}-\frac{n}{k}v^n}{(1+i)^k-1}}.
\]

\subsubsection*{``Each \(1/m\) of a Period''}
\item We shall analyze two kinds of annuity-immediate under this case, that have
different CF patterns (despite having the same CF frequency):
\begin{enumerate}
\item amount of CF increases by \(1/m\) per period:
\begin{center}
\begin{tikzpicture}
\draw[-Latex] (0,0) -- (12,0);
\draw[fill] (0,0) circle [radius=0.5mm];
\draw[fill] (1,0) circle [radius=0.5mm];
\draw[fill] (2,0) circle [radius=0.5mm];
\draw[fill] (3,0) circle [radius=0.5mm];
\draw[fill] (4,0) circle [radius=0.5mm];
\draw[fill] (7,0) circle [radius=0.5mm];
\draw[fill] (8,0) circle [radius=0.5mm];
\draw[fill] (9,0) circle [radius=0.5mm];
\draw[fill] (10,0) circle [radius=0.5mm];
\node[] () at (0,-0.5) {0};
\node[] () at (2,-0.5) {1};
\node[] () at (4,-0.5) {2};
\node[] () at (5.5,-0.5) {\(\cdots\)};
\node[] () at (8,-0.5) {\(n-1\)};
\node[] () at (10,-0.5) {\(n\)};
\node[] () at (1,0.5) {\(+\frac{1}{m}\)};
\node[] () at (2,0.5) {\(+\frac{1}{m}\)};
\node[] () at (3,0.5) {\(+\frac{2}{m}\)};
\node[draw, brown] () at (3,1.2) {\(\uparrow\frac{1}{m}\)};
\node[] () at (4,0.5) {\(+\frac{2}{m}\)};
\node[] () at (5.5,0.5) {\(\cdots\)};
\node[draw, brown] () at (7,1.2) {\(\uparrow\frac{1}{m}\)};
\node[] () at (7,0.5) {\(+\frac{n-1}{m}\)};
\node[] () at (8,0.5) {\(+\frac{n-1}{m}\)};
\node[draw, brown] () at (9,1.2) {\(\uparrow\frac{1}{m}\)};
\node[] () at (9,0.5) {\(+\frac{n}{m}\)};
\node[] () at (10,0.5) {\(+\frac{n}{m}\)};
\end{tikzpicture}
\end{center}
\item \label{it:1m-annuity-incr-each-1m} amount of CF increases by \(1/m^2\) per \(1/m\) of a period:
\begin{center}
\begin{tikzpicture}
\draw[-Latex] (0,0) -- (12,0);
\draw[fill] (0,0) circle [radius=0.5mm];
\draw[fill] (1,0) circle [radius=0.5mm];
\draw[fill] (2,0) circle [radius=0.5mm];
\draw[fill] (3,0) circle [radius=0.5mm];
\draw[fill] (4,0) circle [radius=0.5mm];
\draw[fill] (7,0) circle [radius=0.5mm];
\draw[fill] (8,0) circle [radius=0.5mm];
\draw[fill] (9,0) circle [radius=0.5mm];
\draw[fill] (10,0) circle [radius=0.5mm];
\node[] () at (0,-0.5) {0};
\node[] () at (2,-0.5) {1};
\node[] () at (4,-0.5) {2};
\node[] () at (5.5,-0.5) {\(\cdots\)};
\node[] () at (8,-0.5) {\(n-1\)};
\node[] () at (10,-0.5) {\(n\)};
\node[] () at (1,0.5) {\(+\frac{1}{m^2}\)};
\node[] () at (2,0.5) {\(+\frac{2}{m^2}\)};
\node[draw, brown, font=\small] () at (2,1.2) {\(\uparrow\frac{1}{m^{2}}\)};
\node[] () at (3,0.5) {\(+\frac{3}{m^{2}}\)};
\node[draw, brown, font=\small] () at (3,1.2) {\(\uparrow\frac{1}{m^{2}}\)};
\node[] () at (4,0.5) {\(+\frac{4}{m^{2}}\)};
\node[draw, brown, font=\small] () at (4,1.2) {\(\uparrow\frac{1}{m^{2}}\)};
\node[] () at (6,0.5) {\(\cdots\)};
\node[font=\small] () at (7,0.5) {\(+\frac{mn-3}{m^{2}}\)};
\node[draw, brown, font=\small] () at (7,1.2) {\(\uparrow\frac{1}{m^{2}}\)};
\node[font=\small] () at (8,0.5) {\(+\frac{mn-2}{m^{2}}\)};
\node[draw, brown, font=\small] () at (8,1.2) {\(\uparrow\frac{1}{m^{2}}\)};
\node[font=\small] () at (9,0.5) {\(+\frac{mn-1}{m^{2}}\)};
\node[draw, brown, font=\small] () at (9,1.2) {\(\uparrow\frac{1}{m^{2}}\)};
\node[font=\small] () at (10,0.5) {\(+\frac{mn}{m^{2}}\)};
\node[draw, brown, font=\small] () at (10,1.2) {\(\uparrow\frac{1}{m^{2}}\)};
\end{tikzpicture}
\end{center}
\end{enumerate}

\item \label{it:pv-1m-annuity-incr-each-per}
For the first kind where the amount of CF increases by \(1/m\) per
period, we can perform a ``shifting'' trick to derive its PV formula. First we
``shift'' the CFs in the following way:
\begin{center}
\begin{tikzpicture}
\draw[-Latex] (0,0) -- (12,0);
\draw[fill] (0,0) circle [radius=0.5mm];
\draw[fill] (1,0) circle [radius=0.5mm];
\draw[fill] (2,0) circle [radius=0.5mm];
\draw[fill] (3,0) circle [radius=0.5mm];
\draw[fill] (4,0) circle [radius=0.5mm];
\draw[fill] (7,0) circle [radius=0.5mm];
\draw[fill] (8,0) circle [radius=0.5mm];
\draw[fill] (9,0) circle [radius=0.5mm];
\draw[fill] (10,0) circle [radius=0.5mm];
\node[] () at (0,-0.5) {0};
\node[] () at (2,-0.5) {1};
\node[] () at (4,-0.5) {2};
\node[] () at (5.5,-0.5) {\(\cdots\)};
\node[] () at (8,-0.5) {\(n-1\)};
\node[] () at (10,-0.5) {\(n\)};
\node[] () at (1,0.5) {\(+\frac{1}{m}\)};
\node[] () at (2,0.5) {\(+\frac{1}{m}\)};
\node[] () at (3,0.5) {\(+\frac{2}{m}\)};
\node[] () at (4,0.5) {\(+\frac{2}{m}\)};
\node[] () at (5.5,0.5) {\(\cdots\)};
\node[] () at (7,0.5) {\(+\frac{n-1}{m}\)};
\node[] () at (8,0.5) {\(+\frac{n-1}{m}\)};
\node[] () at (9,0.5) {\(+\frac{n}{m}\)};
\node[] () at (10,0.5) {\(+\frac{n}{m}\)};

\draw[-Latex] (0,-2) -- (12,-2);
\draw[fill] (0,-2) circle [radius=0.5mm];
\draw[fill] (2,-2) circle [radius=0.5mm];
\draw[fill] (4,-2) circle [radius=0.5mm];
\draw[fill] (8,-2) circle [radius=0.5mm];
\draw[fill] (10,-2) circle [radius=0.5mm];
\node[] () at (0,-2.5) {0};
\node[] () at (2,-2.5) {1};
\node[] () at (4,-2.5) {2};
\node[] () at (5.5,-2.5) {\(\cdots\)};
\node[] () at (8,-2.5) {\(n-1\)};
\node[] () at (10,-2.5) {\(n\)};
\node[] () at (2,-1.5) {\(+\sx{\angl{1}}[(m)]\)};
\node[] () at (4,-1.5) {\(+2\sx{\angl{1}}[(m)]\)};
\node[] () at (5.5,-1.5) {\(\cdots\)};
\node[] () at (8,-1.5) {\(+(n-1)\sx{\angl{1}}[(m)]\)};
\node[] () at (10,-1.5) {\(+n\sx{\angl{1}}[(m)]\)};

\draw[-Latex, violet] (1.2,0.3) -- (2,-1.3);
\draw[-Latex, violet] (2.0,0.3) -- (2,-1.3);
\draw[-Latex, violet] (3.2,0.3) -- (4,-1.3);
\draw[-Latex, violet] (4.0,0.3) -- (4,-1.3);
\draw[-Latex, violet] (7.2,0.3) -- (8,-1.3);
\draw[-Latex, violet] (8.0,0.3) -- (8,-1.3);
\draw[-Latex, violet] (9.2,0.3) -- (10,-1.3);
\draw[-Latex, violet] (10.0,0.3) -- (10,-1.3);

\node[] () at (6,1.5) {Original};
\node[] () at (6,-3) {Shifted};
\end{tikzpicture}
\end{center}
From this, we know that its PV is
\[
\Ia_{\angl{n}}^{(m)}=\sx{\angl{1}}[(m)]\Ia_{\angl{n}}
=\frac{(1+i)-1}{i^{(m)}}\Ia_{\angl{n}}=\frac{i}{i^{(m)}}\frac{\ax**{\angl{n}}-nv^n}{i}
=\boxed{\frac{\ax**{\angl{n}}-nv^n}{i^{(m)}}}.
\]
\begin{mnemonic}
The expression is \(\Ia_{\angl{n}}\) with \(i\to i^{(m)}\).
\end{mnemonic}

\item \label{it:pv-1m-annuity-incr-each-1m}
For the second kind where the amount of CF increases by \(1/m^2\) per \(1/m\)
of a period, we can use \(j\)-method to derive its PV formula (the rate \(j\)
for \(1/m\) of a period can be found by solving \(1+j=(1+i)^{1/m}\) for \(j\)):
\[
\text{PV}=(I^{(m)}a)_{\angl{n}}^{(m)}=
\frac{1}{m^2}\Ia_{\angl{mn}j}
=\frac{1}{m^2}\cdot\frac{\overbrace{\ax**{\angl{mn}j}}^{m\ax**{\angl{n}i}[(m)]}-mn(1+j)^{-mn}}{\underbrace{j}_{i^{(m)}/m}}
=\boxed{\frac{\ax**{\angl{n}}[(m)]-nv^n}{i^{(m)}}}.
\]
\begin{mnemonic}
The expression is \(\Ia_{\angl{n}}^{(m)}\) with \(\ax**{\angl{n}}\to \ax**{\angl{n}}[(m)]\).
\end{mnemonic}
\end{enumerate}
\subsection{Continuous Annuities With CFs Varying Continuously}
\begin{enumerate}
\item In \cref{subsect:cts-annuity}, we assume that the rate of cash inflows
for \(n\)-period continuous annuity stays at a constant 1. Here we consider a
more general case where the rate can vary as a function of time \(t\) (``vary
continuously'').
\item We first consider a special case where the rate is \(t\) at time \(t\)
and we have a constant force of interest \(\delta\). We can observe that this
case can be regarded as the limit of \labelcref{it:1m-annuity-incr-each-1m} as
\(m\to\infty\). Hence, its PV is
\[
(\bar{I}\bar{a})_{\angl{n}}=\lim_{m\to \infty}(I^{(m)}a)_{\angl{n}}^{(m)}
=\lim_{m\to \infty}\frac{\ax**{\angl{n}}[(m)]-nv^n}{i^{(m)}}
=\frac{\ax*{\angl{n}}-nv^n}{\delta}.
\]
\item The general case is that the rate is \(g(t)\) at time \(t\) for some
function \(g\), and we have a varying force of interest \(\delta_t\) (as a
function of \(t\)). In this case, the PV is
\[
\int_{0}^{n}g(t)a^{-1}(t)\dd{t}
=\int_{0}^{n}g(t)\exp(-\int_{0}^{t}\delta_s\dd{s})\dd{t}
\]
(which may be seen as following from the definition).

\begin{intuition}
In each infinitesimal time interval \([t,t+\dd{t}]\), the PV of the ``stream''
of cash inflows during the interval (total amount: \(\text{rate}\times\text{time length}=g(t)\dd{t}\)) is
\[{\color{brown}\text{total amount}}\times{\color{violet}\text{discount factor}}
={\color{brown}g(t)}{\color{violet}a^{-1}(t)}{\color{brown}\dd{t}}.\]
``Summing'' all the PVs up then gives the total PV:
\[
\int_{0}^{n}g(t)a^{-1}(t)\dd{t}.
\]
\end{intuition}
\end{enumerate}
\subsection{Annuities With CFs Varying in Geometric Sequence}
\begin{enumerate}
\item Apart from arithmetic sequence, the CFs in an annuity can also vary in a
\emph{geometric sequence}. But in such case, for most calculations, using the
\emph{geometric series formula} would be sufficient:
\[
a+ar+ar^2+\dotsb+ar^{n-1}
=
\begin{cases}
\displaystyle\frac{a(1-r^n)}{1-r}&\text{if \(r\ne 1\)};\\
an&\text{if \(r=1\)}.
\end{cases}
\]
(One just needs to be careful about what the common ratio \(r\) is.)

\item \label{it:pv-gs-annuity-imm}
But for the sake of completeness, here we will derive some formulas in
this case. Consider an \(n\)-period annuity-immediate where the cash inflows at
time \(1,2,\dotsc,n\) are \(1,(1+k),\dotsc,(1+k)^{n-1}\) with \(k> -1\) (to
avoid negative cash inflow) respectively. Then, its PV is \[
v+(1+k)v^2+\dotsb+(1+k)^{n-1}v^n=
\begin{cases}
nv&\text{if \(i=k\)};\\
\displaystyle \frac{v[1-(v(1+k))^{n}]}{1-v(1+k)}&\text{if \(i\ne k\)}.
\end{cases}
\]
\begin{center}
\begin{tikzpicture}
\draw[-Latex] (0,0) -- (12,0);
\draw[fill] (0,0) circle [radius=0.5mm];
\draw[fill] (2,0) circle [radius=0.5mm];
\draw[fill] (4,0) circle [radius=0.5mm];
\draw[fill] (8,0) circle [radius=0.5mm];
\draw[fill] (10,0) circle [radius=0.5mm];
\node[] () at (0,-0.5) {0};
\node[] () at (2,-0.5) {1};
\node[] () at (4,-0.5) {2};
\node[] () at (6,-0.5) {\(\cdots\)};
\node[] () at (8,-0.5) {\(n-1\)};
\node[] () at (10,-0.5) {\(n\)};
\node[] () at (2,0.5) {\(+1\)};
\node[] () at (4,0.5) {\(+(1+k)\)};
\node[] () at (8,0.5) {\(+(1+k)^{n-2}\)};
\node[] () at (10,0.5) {\(+(1+k)^{n-1}\)};
\end{tikzpicture}
\end{center}
\item For \emph{perpetuity-immediate} whose cash inflows at time
\(1,2,\dotsc,\) are \(1,(1+k),\dotsc\) (assuming \(i\ne k\) and \(k> -1\))
respectively, its PV is
\[
\lim_{n\to \infty}\frac{v[1-(v(1+k))^{n}]}{1-v(1+k)}
=\frac{v}{1-v(1+k)}
=\frac{1}{1+i-1-k}
=\frac{1}{i-k}.
\]
\end{enumerate}
\subsection{Rainbow Immediate}
\begin{enumerate}
\item Starting from here, we will discuss some annuities that are ``exotic'' in
terms of CF pattern. The first one is known as \emph{rainbow immediate}, whose
CF pattern looks like a ``rainbow'':

\begin{center}
\begin{tikzpicture}
\draw[-Latex] (0,0) -- (12,0);
\foreach \x in {0,...,9} {
\draw[fill] (\x,0) circle [radius=0.5mm];
\node[] () at (\x,-0.5) {\x};}
\draw[opacity=0.4, red, line width=0.2cm, line cap=round, rounded corners=0.5cm] (1,0.5) -- (5,2.6) -- (9,0.5);
\draw[opacity=0.4, orange!50!yellow, line width=0.2cm, line cap=round, rounded corners=0.5cm] (2,0.5) -- (5,2.1) -- (8,0.5);
\draw[opacity=0.4, green, line width=0.2cm, line cap=round, rounded corners=0.5cm] (3,0.5) -- (5,1.6) -- (7,0.5);
\draw[opacity=0.4, blue, line width=0.2cm, line cap=round, rounded corners=0.5cm] (4,0.5) -- (5,1.1) -- (6,0.5);
\draw[opacity=0.4, violet, line width=0.2cm, line cap=round, rounded corners=0.5cm] (4.7,0.5) -- (5,0.7) -- (5.3,0.5);
\foreach \x in {1,...,9}\node[] () at (\x,0.5) {\(+1\)};
\foreach \x in {2,...,8}\node[] () at (\x,1) {\(+1\)};
\foreach \x in {3,...,7}\node[] () at (\x,1.5) {\(+1\)};
\foreach \x in {4,...,6}\node[] () at (\x,2) {\(+1\)};
\node[] () at (5,2.5) {\(+1\)};
\end{tikzpicture}
\end{center}
\item More precisely, an \defn{\(2n-1\))-period rainbow immediate} is an
\(2n-1\)-period annuity-immediate whose CFs at time
\(1,\dotsc,n,n+1,\dotsc,2n-1\) are \(1,\dotsc,n,n-1,\dotsc,1\) respectively.
\item \label{it:rainbow-imm-pv-split-shift}
There are two ways to derive its PV formula. The first way is the
``splitting'' and ``shifting'' approach:
\begin{center}
\begin{tikzpicture}
\draw[-Latex] (0,0) -- (12,0);
\foreach \x in {0,...,9} {
\draw[fill] (\x,0) circle [radius=0.5mm];
\node[] () at (\x,-0.5) {\x};}
\draw[fill, opacity=0.4, orange!50!yellow] (0.8,0.3) -- (5.2,2.7) -- (5.2,0.3) -- cycle;
\draw[-Latex, violet] (3.5,1.2) to[bend right] node[bend right, auto, swap]{shift} (0,0.8);
\node[orange] () at (0,0.5) {\(+\Ia_{\angl{n}}\)};
\draw[fill, opacity=0.4, blue!50!white] (5.8,0.3) -- (9.2,0.3) -- (5.8,2.2) -- cycle;
\draw[-Latex, violet] (7.5,1) to[bend right] node[bend right, auto, swap]{shift} (5,3);
\node[blue] () at (5,3.5) {\(+\Da_{\angl{n-1}}\)};
\foreach \x in {1,...,9}\node[] () at (\x,0.5) {\(+1\)};
\foreach \x in {2,...,8}\node[] () at (\x,1) {\(+1\)};
\foreach \x in {3,...,7}\node[] () at (\x,1.5) {\(+1\)};
\foreach \x in {4,...,6}\node[] () at (\x,2) {\(+1\)};
\node[] () at (5,2.5) {\(+1\)};
\end{tikzpicture}
\end{center}
From this, the PV is
\[
\Ia_{\angl{n}}+v^n\Da_{\angl{n-1}}.
\]
\item \label{it:rainbow-imm-pv-diag-horizon}
Another way is to have a ``diagonal \& horizontal'' view and ``shifting'':
\begin{center}
\begin{tikzpicture}
\draw[-Latex] (0,0) -- (12,0);
\foreach \x in {0,...,9} {
\draw[fill] (\x,0) circle [radius=0.5mm];
\node[] () at (\x,-0.5) {\x};}
\draw[opacity=0.4, red, line width=0.2cm, line cap=round] (1,0.5) -- (5.2,2.5);
\draw[opacity=0.4, orange!50!yellow, line width=0.2cm, line cap=round, rounded corners=0.5cm] (2,0.5) -- (5,2.1) -- (6.2,2.1);
\draw[opacity=0.4, green, line width=0.2cm, line cap=round, rounded corners=0.5cm] (3,0.5) -- (5,1.6) -- (7.2,1.6);
\draw[opacity=0.4, blue, line width=0.2cm, line cap=round, rounded corners=0.5cm] (4,0.5) -- (5,1.1) -- (8.2,1.1);
\draw[opacity=0.4, violet, line width=0.2cm, line cap=round] (5,0.5) -- (9.2,0.5);
\foreach \x in {1,...,9}\node[] () at (\x,0.5) {\(+1\)};
\foreach \x in {2,...,8}\node[] () at (\x,1) {\(+1\)};
\foreach \x in {3,...,7}\node[] () at (\x,1.5) {\(+1\)};
\foreach \x in {4,...,6}\node[] () at (\x,2) {\(+1\)};
\node[] () at (5,2.5) {\(+1\)};

\draw[-Latex] (0,-2) -- (12,-2);
\foreach \x in {0,...,9} {
\draw[fill] (\x,-2) circle [radius=0.5mm];
\node[] () at (\x,-2.5) {\x};}
\node[red] () at (1,-1.5) {\(+\ax**{\angl{n}}\)};
\node[orange!50!yellow] () at (2,-1.5) {\(+\ax**{\angl{n}}\)};
\node[green!80!black] () at (3,-1.5) {\(+\ax**{\angl{n}}\)};
\node[blue] () at (4,-1.5) {\(+\ax**{\angl{n}}\)};
\node[violet] () at (5,-1.5) {\(+\ax**{\angl{n}}\)};

\draw[-Latex, brown] (1.5,0.7) -- (1.1,-1.1);
\draw[-Latex, brown] (2.5,0.7) -- (2.1,-1.1);
\draw[-Latex, brown] (3.5,0.7) -- (3.1,-1.1);
\draw[-Latex, brown] (4.5,0.7) -- (4.1,-1.1);
\draw[-Latex, brown] (5.5,0.5) -- (5.1,-1.1);
\node[brown] () at (6,-1.1) {shift};
\end{tikzpicture}
\end{center}
From this, we know that the PV is \(\ax**{\angl{n}}\ax{\angl{n}}\).
\end{enumerate}
\subsection{Paused Rainbow Immediate}
\begin{enumerate}
\item The next annuity is \emph{paused rainbow immediate}, whose
CF pattern looks like a ``rainbow'' with a ``pause'' at the middle:
\begin{center}
\begin{tikzpicture}
\draw[-Latex] (0,0) -- (12,0);
\foreach \x in {0,...,10} {
\draw[fill] (\x,0) circle [radius=0.5mm];
\node[] () at (\x,-0.5) {\x};}
\draw[opacity=0.4, red, line width=0.2cm, line cap=round, rounded corners=0.5cm] (1,0.5) -- (5,2.6) -- (6,2.6) -- (10,0.5);
\draw[opacity=0.4, orange!50!yellow, line width=0.2cm, line cap=round, rounded corners=0.5cm] (2,0.5) -- (5,2.1) -- (6,2.1) -- (9,0.5);
\draw[opacity=0.4, green, line width=0.2cm, line cap=round, rounded corners=0.5cm] (3,0.5) -- (5,1.6) -- (6,1.6) -- (8,0.5);
\draw[opacity=0.4, blue, line width=0.2cm, line cap=round, rounded corners=0.5cm] (4,0.5) -- (5,1.1) -- (6,1.1) -- (7,0.5);
\draw[opacity=0.4, violet, line width=0.2cm, line cap=round, rounded corners=0.5cm] (4.7,0.5) -- (5,0.7) -- (6,0.7) -- (6.3,0.5);
\foreach \x in {1,...,10}\node[] () at (\x,0.5) {\(+1\)};
\foreach \x in {2,...,9}\node[] () at (\x,1) {\(+1\)};
\foreach \x in {3,...,8}\node[] () at (\x,1.5) {\(+1\)};
\foreach \x in {4,...,7}\node[] () at (\x,2) {\(+1\)};
\foreach \x in {5,...,6}\node[] () at (\x,2.5) {\(+1\)};
\end{tikzpicture}
\end{center}

\item More precisely, an \defn{\(2n\)-period paused rainbow immediate} is an
\(2n\)-period annuity-immediate whose CFs at time
\(1,\dotsc,n,n+1,\dotsc,2n-1\) are \(1,\dotsc,n,n,\dotsc,1\) respectively.

\item \label{it:paused-rainbow-imm-pv-fmlas}
In a similar manner as
\labelcref{it:rainbow-imm-pv-split-shift,it:rainbow-imm-pv-diag-horizon}, we
can derive two formulas for PV of an \(n\)-period paused rainbow immediate:
\begin{itemize}
\item \(\text{PV}=\Ia_{\angl{n}}+v^n\Da_{\angl{n}}\);
\item \(\text{PV}=\ax**{\angl{n+1}}\ax{\angl{n}}\).
\end{itemize}
\end{enumerate}
\subsection{``Paused-From-Time-\(m\)'' Increasing Annuity}
\begin{enumerate}
\item The last ``exotic'' annuity discussed here is the so-called \(n\)-period
``paused-from-time \(m\)'' increasing annuity (this name is not standard),
where \(m<n, m,n\in\N\).  Simply speaking, it is an \(n\)-period increasing
annuity but the CF stays at \(m\) (``paused'') starting from time \(m\) until the end.
\begin{center}
\begin{tikzpicture}
\draw[-Latex] (0,0) -- (12,0);
\foreach \x in {0,...,9} {
\draw[fill] (\x,0) circle [radius=0.5mm];
}
\node[] () at (0,-0.5) {0};
\node[] () at (4,-0.5) {\(m\)};
\node[] () at (9,-0.5) {\(n\)};
\foreach \x in {1,...,9}\node[] () at (\x,0.5) {\(+1\)};
\foreach \x in {2,...,9}\node[] () at (\x,1) {\(+1\)};
\foreach \x in {3,...,9}\node[] () at (\x,1.5) {\(+1\)};
\foreach \x in {4,...,9}\node[] () at (\x,2) {\(+1\)};
\end{tikzpicture}
\end{center}
\item \label{it:pv-pause-from-m-incr-annuity-split-shift}
To get its PV, we can use the ``splitting'' and ``shifting'' approach:
\begin{center}
\begin{tikzpicture}
\draw[-Latex] (0,0) -- (12,0);
\foreach \x in {0,...,9} {
\draw[fill] (\x,0) circle [radius=0.5mm];
}
\node[] () at (0,-0.5) {0};
\node[] () at (4,-0.5) {\(m\)};
\node[] () at (9,-0.5) {\(n\)};
\draw[fill, opacity=0.4, orange!50!yellow] (0.8,0.3) -- (4.2,2.2) -- (4.2,0.3) -- cycle;
\draw[fill, opacity=0.4, blue!50!white] (4.8,0.3) rectangle (9.2,2.2);
\foreach \x in {1,...,9}\node[] () at (\x,0.5) {\(+1\)};
\foreach \x in {2,...,9}\node[] () at (\x,1) {\(+1\)};
\foreach \x in {3,...,9}\node[] () at (\x,1.5) {\(+1\)};
\foreach \x in {4,...,9}\node[] () at (\x,2) {\(+1\)};
\draw[-Latex, violet] (3.5,1.2) to[bend right] node[bend right, auto, swap]{shift} (0,0.8);
\node[orange] () at (0,0.5) {\(+\Ia_{\angl{m}}\)};
\draw[-Latex, violet] (7.5,1.5) to[bend right] node[bend right, auto, swap]{shift} (4,2.5);
\node[blue] () at (4,3) {\(+m\ax{\angl{n-m}}\)};
\end{tikzpicture}
\end{center}
Hence, the PV is
\[
(I_{\angl{m}}a)_{\angl{n}}=\Ia_{\angl{m}}+v^m(m\ax{\angl{n-m}}).
\]
\begin{note}
``\(I_{\angl{m}}\)'' represents ``increasing for \(m\) periods (up to time \(m\))''.
\end{note}
\end{enumerate}
\subsection{Summary of Tricks and Intuition}
\begin{enumerate}
\item The following are tricks/intuition mentioned for deriving various
formulas, and they are also useful in solving problems sometimes:
\begin{enumerate}
\item ``splitting'' and ``shifting'' (\faIcon{check} useful generally)
\item \(j\)-method (\faIcon{check} ``more frequent''/``less frequent'' case)
\item think in ``infinitesimal'' (\faIcon{check} continuous case)
\item ``horizontal'' and ``diagonal'' views (\faIcon{check} varying CFs case)
\end{enumerate}
\end{enumerate}
\subsection{Varying Effective Interest Rate}
\begin{enumerate}
\item Lastly, we discuss a general setting where the effective interest rate
can vary and compound interest may \emph{not} be assumed throughout. In this
case, we generally need to use ``first principle'' to compute the PV/AV of an
annuity, namely summing the PV/AV of each CF one by one.
\item \label{it:pv-av-annuities-varying-effective}
To be more explicit, we have the following (general) formulas, using the
effective interest rate notation:
\begin{itemize}
\item \(n\)-period annuity-immediate:
\begin{align*}
\text{PV}&=a^{-1}(1)+a^{-1}(2)+\dotsb+a^{-1}(n) \\
&=\boxed{(1+i_1)^{-1}+(1+i_1)^{-1}(1+i_2)^{-1}+\dotsb+(1+i_1)^{-1}\dotsb(1+i_n)^{-1}}.
\end{align*}
and
\begin{align*}
\text{time-\(n\) AV}&=\text{PV}\cdot(1+i_1)(1+i_2)\dotsb(1+i_n) \\
&=\boxed{1+(1+i_{n})+(1+i_{n-1})(1+i_{n})+\dotsb+(1+i_2)\dotsb(1+i_{n})}.
\end{align*}
\item \(n\)-period annuity-due
\begin{align*}
\text{PV}&=a^{-1}(0)+a^{-1}(1)+\dotsb+a^{-1}(n-1) \\
&=\boxed{1+(1+i_1)^{-1}+\dotsb+(1+i_1)^{-1}\dotsb(1+i_{n-1})^{-1}}.
\end{align*}
and
\begin{align*}
\text{time-\(n\) AV}&=\text{PV}\cdot(1+i_1)(1+i_2)\dotsb(1+i_n). \\
&=\boxed{(1+i_{n})+(1+i_{n-1})(1+i_{n})+\dotsb+(1+i_1)\dotsb(1+i_{n})}.
\end{align*}
\end{itemize}
\end{enumerate}
